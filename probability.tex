\section{Pravděpodobnost}

\subsection{Úvod}
Dobrým zdrojom sú materiály z predmetu \verb|IV111| Probability in Computer Science.

Čo sa týka poslednej odrážky v otázke (\uv{aplikace v informace}), tu pôjde
zrejme najmä o~pravdepodobnostné algoritmy, ktoré popisujeme hlavne v~sekcii
{\uv{Algoritmy pro těžké problémy} (čiastočne aj v~sekciách \uv{Složitost} 
a \uv{Grafy a grafové algoritmy}).


\subsubsection*{Znenie otázky}
Pravděpodobnost. Základní pojmy a vlastnosti pravděpodobnosti; 
náhodná proměnná a její charakteristiky (střední hodnota, rozptyl, momenty, ....) 
a jejich použití; binomická, geometrická a další distribuce a jejich charakteristiky. 
Markovova a Čebyševova nerovnost a jejich použití; Černoffovy odhady a jejich použití; 
náhodné, zejména Markovovy, procesy a jejich vlastnosti; entropie a informace 
a jejich vlastnosti; aplikace v informatice.

\subsubsection*{Pojmy}
Náhodný pokus, základný priestor (konečný, spočetný, nespočetný); hod mincou, kockou;
jav (nemožný, istý, elementárny, opačný), de Morganove zákony, nezávislé 
javy (vzájomne, kolektívne), rozklad základného priestoru, 
pravdepodobnosť (diskrétna, spojitá), princíp inklúzie/exklúzie,
pravdepodobnostný priestor (diskrétny, spojitý), $\sigma$-field, 
podmienená pravdepodobnosť, zákon o totálnej pravdepodobnosti,
Bayesove pravidlá; náhodná premenná (diskrétna, spojitá), 
pravdivostná funkcia (distribúcia), distribučná funkcia, 
hustota, marginálna distribúcia, stredná hodnota, momenty,
rozptyl, smerodatná odchýlka, náhodná generujúca funkcia;
Bernoulliho rozdelenie, geometrické rozdelenie, binomické
rozdelenie, rovnomerné rozdelenie, exponenciálne rozdelenie,
normálne rozdelenie; Markovova nerovnosť, korelačný koeficient,
kovariancia, Chebyshevova nerovnosť; moment generujúce funkcie,
Chernoffove odhady; stochastické procesy (konečné, diskrétne,
spojité), Markovov reťazec, matica prechodu, iniciálny vektor,
hitting time, hitting probability, stavy Markovovho
modelu (absorbující, rekurentní, průchozí), ireducibilný Markov Chain (MC),
stacionárna distribúcia, periodický/aperiodický MC;
entropia (Shannon), joint entropy, podmienená entropia, 
entropy chain rule, relatívna entropia, spoločná informácia.

\subsection{Základní pojmy a vlastnosti pravděpodobnosti}

\begin{definition}
Pro daný náhodný experiment je diskrétní pravděpodobnostní prostor trojice
$(\Omega, \mathcal{F}, P)$,
kde
\begin{itemize}
    \item $\Omega$ je množina možných výsledků experimentu,
    \item $\mathcal{F} \subseteq \powerset{\Omega}$ je množina možných události (javov),
    \item $P : \mathcal{F} \to [0,1]$ je pravděpodobnostní funkce, tedy
        pro po párech disjunktní $\{A_i\}_{i=1}^\infty \subseteq \mathcal{F}$ platí
        $P(\bigcup A_i) = \sum P(A_i)$ a $P(\Omega) = 1$.
\end{itemize}
\end{definition}

\begin{example}
    Můžeme například uvažovat experiment hodu dvěma kostkami.
    $\Omega = \{1,\ldots,6\}^2$, pro diskrétní $\Omega$ platí
    $\mathcal{F} = \powerset{\Omega}$ a z~uniformity kostek
    $P((x,y)) = 1/\lvert \Omega \rvert, (x,y) \in \Omega$.
\end{example}

Špeciálne prípady pri javoch sú istý jav (celá množina $\Omega$, určite nastane),
nemožný jav (prázdna množina $\emptyset$, nikdy nenastane),
elementárny jav (napr. to, že pri hode kockou padne práve 1,
neelementárnym javom by bol napr. jav, že padne nepárne číslo,
ktorý je zložený z troch elementárnych javov).

\begin{definition}
	Javy $A_1, \ldots, A_n$ nazývame {\em vzájomne disjunktné}, 
	pokiaľ platí $\forall i \neq j.P(A_i \land A_j)=0$.
	
	Javy $A_1, \ldots, A_n$ nazývame {\em kolektívne vyčerpávajúce}, %spravny termin? 
	pokiaľ platí $P(A_1 \lor \ldots \lor A_j)=1$.
	
	Javy $A_1, \ldots, A_n$ sú {\em rozkladom} základného priestoru,
	pokiaľ sú vzájomne disjunktné a kolektívne vyčerpávajúce.
\end{definition}

\begin{claim}
    Nechť $A$ je událost a $\overline{A}$ její doplněk. Pak
    $P(\overline{A}) = 1 - P(A)$.
\end{claim}

\begin{proof}
    Víme $P(A \cup \overline{A}) = 1$ a $P(A \cap \overline{A}) = 0$,
	tedy $P(A) + P(\overline{A}) - P(A \cap \overline{A}) = P(A) + P(\overline{A}) = 1$.
\end{proof}

\begin{definition}
Podmíněnou pravděpodobnost výskytu události $A$ za předpokladu výskytu
události $B$ definujeme
\[
    P(A \mid B) = \frac{P(A \cap B)}{P(B)}
\]
\end{definition}

Jednoduchou intuici za definicí nabízí Vennův diagram.

\begin{theorem}[Totální pravděpodobnosti]
	Nech javy $B_i$ sú rozkladom základného priestoru. Potom
    \[
        P(A) = \sum_i P(A \cap B_i) = \sum_i P(A \mid B_i) P(B_i)
    \]
\end{theorem}

\begin{example}
    Žárovky X svítí přes 5000 hodin v~99 \% případů.
    Žárovky Y svítí přes 5000 hodin v~95 \% případů.
    Žárovky X tvoří 60 \% dostupných žárovek.
    Žárovky Y tvoří 40 \% dostupných žárovek.
    Jaká je pravděpodobnost, že náhodně zvolená žárovka
    bude svítit přes 5000 hodin?
    \[
{\displaystyle {\begin{aligned}\Pr(A)&=\Pr(A\mid B_{X})\cdot \Pr(B_{X})+\Pr(A\mid B_{Y})\cdot \Pr(B_{Y})\\[4pt]&={99 \over 100}\cdot {6 \over 10}+{95 \over 100}\cdot {4 \over 10}={{594+380} \over 1000}={974 \over 1000}\end{aligned}}}
\]
\end{example}

\begin{theorem}[Bayes]
    Pokud $P(B) \neq 0$, pak
    \[
    P(A \mid B) = \frac{P(B \mid A)P(A)}{P(B)}
    \]
\end{theorem}

Věta vyplývá z~jednoduché úpravy definice podmíněné pravděpodobnosti.

\begin{example}
    % https://www.wikiskripta.eu/w/Bayesova_v%C4%9Bta
Předpokládejme, že máme školu s~60~\% chlapců
a 40~\% dívek. Všichni chlapci nosí kalhoty.
Z~dívek nosí kalhoty polovina. Pozorovatel vidí z~dálky studenta v kalhotách.
Jaká je pravděpodobnost, že tento student je dívka?
Jako událost G označíme, že pozorovaný student je dívka. Jako událost T,
že pozorovaný student nosí kalhoty. Chceme vypočíst $P(G \mid T)$.
\[
    P(G \mid T)
    = \frac{P(T \mid G)P(G)}{P(T)}
    = \frac{1/2 \cdot 4/10}{6/10 \cdot 1 + 4/10 \cdot 1/2}
	= \frac{1}{4}
\]
\end{example}

Keď je základný priestor $S$ spojitý, dá sa miesto neho
v určitých prípadoch použiť $\sigma$-field.

\begin{definition}
	Nech $S$ je spojitý základný priestor. Potom $\mathscr{F} \subseteq 2^S$
	je $\sigma$-field, pokiaľ
	\begin{itemize}
		\item $\emptyset \in \mathscr{F}$
		\item $A,B \in \mathscr{F} \Rightarrow A \cup B \in \mathscr{F}$
		\item $A \in \mathscr{F} \Rightarrow \overline{A} \in \mathscr{F}$
	\end{itemize}
\end{definition}

Poznamenajme, že vďaka De Morganovym zákonom je $\sigma$-field uzavretý
aj na prienik.

\subsection{Náhodná proměnná a její charakteristiky}

\begin{definition}
    Pro (diskrétní) pravděpodobnostní prostor $(\Omega, \mathcal{F}, P)$
    nazveme {\em (diskrétní) náhodnou proměnnou} funkci $X : \Omega \to \mathbb{R}$.
\end{definition}

\begin{example}
    Například pro
    $(\{1,\ldots,6\}, \powerset{\{1,\ldots,6\}}, P(x) = 1/6)$
    můžeme definovat náhodnou proměnnou
    $X(x) = 0, x = 1,\ldots,5, X(6) = 1$.
\end{example}

\begin{definition}
    Pravdepodobnostná funkcia (alebo pravdepodobnostná distribúcia)
	náhodnej premennej $X$ je funkcia $p_X: \mathbb{R} \to [0,1]$ definovaná ako
	\[
    p_X(x) = P(X=x) = \sum_{s:X(s)=x}P(s)
    \]
\end{definition}

\begin{definition}
    Distribučná funkcia
	náhodnej premennej $X$ je funkcia $F_X: \mathbb{R} \to [0,1]$ definovaná ako
	\[
    F_X(x) = P(X\leq x) = \sum_{t \leq x}p_X(t)
    \]
\end{definition}

Pre distribučnú funkciu platí:
\begin{enumerate}
	\item pre $-\infty < x < \infty$ platí $0 \leq F(x) \leq 1$
	\item F(x) je neklesajúca funkcia
	\item $lim_{x \to -\infty}F(x) = 0$ a $lim_{x \to \infty}F(x) = 1$
	\item TODO
\end{enumerate}

V prípade spojitej distribúcie nemá zmysel hovoriť o pravdepodobnostnej
funkcii. V tomto prípade sa používa {\em hustota}, ktorá je definovaná
ako derivácia distribučnej funkcie.

\begin{definition}[Očekávaná (střední) hodnota]
    Nechť $(\Omega, \mathcal{F}, P)$ je diskrétní pravděpodobnostní
    prostor a $X$ náhodná proměnná. {\em Očekávaná hodnota} $X$
    je
    \[ \Expect(X) = \sum_{x \in \Image(X)} x \cdot P(X = x) \]
\end{definition}

\begin{theorem}
    Nechť $(X_i)_{i=1}^{n}$, $n \in \mathbb{N}$, je posloupnost
    náhodných proměnných \linebreak s~konečnou očekávanou hodnotou, potom
    \[
        E \big ( \sum X_i \big ) = \sum E(X_i)
    \]
\end{theorem}

\begin{proof}
    Indukcí k~počtu sčítanců. Bázový případ je triviální. Nechť $X$ je
    náhodná proměnná a $Y$ je součtem nenulového počtu náhodných proměnných.

    \begin{align*}
        E(X + Y) &= \sum_i \sum_j (i+j) P((X = i) \cap (Y = j)) \\
                 &= \sum_i \sum_j i P((X = i) \cap (Y = j))
                  + \sum_i \sum_j j P((X = i) \cap (Y = j)) \\
                 &= \sum_i i \sum_j P((X = i) \cap (Y = j))
                  + \sum_j j \sum_i P((X = i) \cap (Y = j)) \\
                 &= \sum_i i P(X = i)
                  + \sum_j j P(Y = j) \\
                 &= E(X) + E(Y)
                   \qedhere
    \end{align*}
\end{proof}

\begin{theorem}
    \[
        E(cX) = cE(X)
    \]
\end{theorem}

\begin{proof}
    Pro $c = 0$ je tvrzení zřejmé. Pro $c \neq 0$ platí

    \begin{align*}
        E(cX) &=   \sum_i i P(cX = i) \\
              &= c \sum_i (i/c) P(X = i/c) \\
              &= c \sum_j j P(X = j) \\
              &= c E(X)
                \qedhere
    \end{align*}
\end{proof}

\begin{definition}
    Proměnné $X,Y$ nazveme {\em nezávislé} pokud pro všechna $x,y$
    \[
        P(X = x \land Y = y) = P(X = y) P(Y = y)
    \]
\end{definition}

Alternativně $P(X = x \mid Y = y) = P(X = x)$ pro každé $x,y$. (Tady by
bylo dobré si rozmyslet, proč tomu tak je.)

\begin{definition}[Momenty]
    Pre náhodnú premennú $X$ je $k$-ty {\em moment}
	definovaný ako $\Expect(X^k)$ a $k$-ty {\em centrálny moment}
	ako $\Expect((X-\Expect(X))^k)$.
\end{definition}

\begin{definition}[Rozptyl, variance]
    Pro náhodnou proměnnou $X$ je {\em rozptyl}
    \[ \Var(X) = \Expect( (X - \Expect(X))^2 ) \]
\end{definition}

Na rozptyl se tedy můžeme dívat jako na druhou mocninu očekávané
vzdálenosti od očekávané hodnoty. Druhá odmocnina rozptylu je
{\em standardní odchylka}.

Všimnime si, že rozptyl je druhým centrálnym momentom.

Následující tvrzení se snadným důkazem dává užitečný způsob výpočtu
variance.

\begin{claim}
    $ Var(X) = \Expect( (X - \Expect(X))^2 ) = \Expect(X^2) - \Expect(X)^2 $
\end{claim}

\begin{definition}[Kovariance]
    $ Cov(X,Y) = E((X - E(X))(Y - E(Y))) $
\end{definition}

Na kovarianci se můžeme dívat jako na míru společné variance dvou
proměnných.

\begin{claim}
    $Var(X + Y) = Var(X) + Var(Y) + 2 Cov(X,Y)$
\end{claim}

\begin{theorem}
    Jsou-li $X, Y$ nezávislé náhodné proměnné, potom
    \[ E(X \cdot Y) = E(X) \cdot E(Y) \]
\end{theorem}


\begin{corollary}
    Jsou-li $X, Y$ nezávislé náhodné proměnné, pak $Cov(X,Y) = 0$.
\end{corollary}

\begin{proof}
    Jsou-li $X, Y$ nezávislé, potom i $X - E(X)$ a $Y - E(Y)$ jsou
    nezávislé. Můžeme tedy použít větu o násobení nezávislých
    očekávaných hodnot. Potom $E(Z - E(Z)) = E(Z) - E(E(Z)) = 0$.
\end{proof}

\subsection{Distribuce}

\begin{definition}[Uniformní rozdělení]
    Náhodná proměnná $X$ má {\em uniformní distribuci},
    pokud pro všech $K$ hodnot $k$ platí
    \[
        P(X = k) = 1/K
    \]
\end{definition}

\begin{definition}[Bernoulliho rozdělení]
    Náhodná proměnná $X$ má {\em Bernoulliho distribuci} s~parametrem
    $p$, pokud
    \begin{align*}
        P(X = 1) &= p  \\
        P(X = 0) &= 1 - p
    \end{align*}
\end{definition}

Z~definice vyplývá $E(X) = 1 \cdot p + 0 \cdot (1-p) = p$.

\begin{example}
    Príkladom náhodnej premennej s Bernoulliho distribúciou
	je hod nevyváženou mincou, kde hlava padne s pravdepodobnosťou $p$.
\end{example}

\begin{definition}[Binomické rozdělení]
    Náhodná proměnná $X$ má {\em binomickou distribuci} s parametry $(n, p)$,
    pokud pro $0 \leq k \leq n$ platí
\[
    P(X = k) = {{n} \choose {k}} p^k (1 - p)^{n - k}
\]
\end{definition}

\begin{theorem}
    \[
        X \sim Bin(n,p) \implies E(X) = np
    \]
\end{theorem}

\begin{proof}
    Uvažujme $n$ náhodných proměnných $X_i$ s~Bernoulliho distribucí
    s~parametrem $p$.
    Zřejmě $X \sim \sum X_i$, potom
    \[
        E(X) = E( \sum_{i = 1}^{n} X )
             = \sum_{i = 1}^{n} E(X)
             = \sum_{i = 1}^{n} p
             = n p
    \]
\end{proof}

\begin{example}
    Příkladem náhodné proměnné s~binomickým rozdělením je
    počet hlav, které padnou při $n$ hodech mincí ($p = 1/2$).
    Podle věty získáme očekávanou hodnotu $n/2$.
\end{example}

\begin{definition}[Geometrické rozdělení]
    Náhodná proměnná $X$ má {\em geometrickou distribuci} s~parametrem
    $p$, pokud pro $1 \leq k$ platí
\[
    P(X = k) = (1-p)^{k-1} p
\]
\end{definition}

\begin{claim}
    Geometrické rozdělení nemá paměť, tedy
    \[
        P(X = n + k \mid X > k) = P(X = n)
    \]
\end{claim}

\begin{theorem}
    Nechť $X$ je náhodná proměnná s~náhodným rozdělením s~parametrem
    $p$, potom
    \[
        E(X) = \frac{1}{p}
    \]
\end{theorem}

\begin{example}
    Novomanželé si plánují pořizovat děti, dokud se jim nenarodí dcera.
    Pravděpodobnost, že se jim narodí jediné dítě (a tedy dcera) je $1/2$.
    Pravděpodobnost, že se jim narodí dvě děti je $1/4$.
    Pravděpodobnost, že se jim narodí tři děti je $1/8$.
    Očekávaný počet dětí je $2$, tedy jeden syn a jedna dcera.

    Tento příklad také demonstruje, že geometrické rozdělení nemá paměť.
    Je jedno kolik dětí se dosud narodilo, pravděpodobnost narození
    dalších to nijak neovlivňuje.
\end{example}

\begin{definition}[Spojité uniformné rozdelenie]
    Spojitá náhodná premenná $X$ má {\em uniformnú distribúciu},
    pokiaľ pre jej hustotu platí
    \[
        f(x) = 
			\begin{cases}
				\frac{1}{b-a} &\quad x \text{ leží v intervale }[a,b]\\
				0 &\quad\text{inak}
			\end{cases}
    \]
\end{definition}

Pre distribučnú funkciu spojitej uniformnej náhodnej premennej teda platí
	\[
        F(x) = 
			\begin{cases}
				0 &\quad x < a \\
				\frac{x-a}{b-a} &\quad x \text{ leží v intervale }[a,b)\\
				1 &\quad x \geq b
			\end{cases}
    \]

\begin{definition}[Exponenciálne rozdelenie]
    Spojitá náhodná premenná $X$ má {\em exponenciálnu distribúciu}
    s parametrom $\lambda$, pokiaľ pre jej hustotu platí
    \[
        f(x) = \lambda e^{-\lambda x}
    \]
\end{definition}

Pre distribučnú funkciu exponenciálnej náhodnej premennej platí
	\[
        F(x) = 1-e^{-\lambda x}
    \]

\begin{definition}[Normálne rozdelenie]
    Spojitá náhodná premenná $X$ má {\em normálnu distribúciu}
    s parametrami $\mu$ (priemer) a $\sigma$ (rozptyl), pokiaľ pre jej hustotu platí
    \[
        f(x) = \frac{1}{\sqrt{2 \pi \sigma^2}}e^{-\frac{(x-\mu)^2}{2\sigma^2}}
    \]
\end{definition}

Pre distribučnú funkciu normálnej náhodnej premennej platí
	\[
        F(x) = \frac{1}{2} \left( 1+ \text{erf}\left(\frac{x-\mu}{\sigma \sqrt{2}}\right) \right),
    \]
kde erf značí error function.

\subsection{Markovova a Čebyševova nerovnost}

\begin{theorem}[Markovova nerovnost]
    Nechť $X$ je náhodná proměnná, která nabývá pouze nezáporných
    hodnot. Potom pro všechna $a > 0$,
    \[
       P(X \geq a) \leq \frac{E(X)}{a}
\]
\end{theorem}

\begin{proof}
    Pro $a \geq 0$ mějme
    \[
        I =
        \begin{cases}
            1, \quad \text{pokud } X \geq a, \\
            0, \quad \text{jinak.}
        \end{cases}
    \]
    Protože $I$ má Bernoulliho rozdělení, tak platí $E(I) = P(I = 1) = P(X \geq a)$.
    Navíc protože $X \geq 0$, tak platí $I \leq \frac{X}{a}$, na což
    aplikujeme $E$ a získáme
    \[
        P(X \geq a) = E(I) \leq E \left(\frac{X}{a} \right) = \frac{E(X)}{a}
        \qedhere
    \]
\end{proof}

\begin{example}
    Předpokládejme $n$-krát opakovaný experiment hodu mincí,
    a $X_i = 1$ padla-li hlava, $X_i = 0$ jinak.
    Potom $X = \sum X_i$ značí počet padlých hlav. Nyní můžeme snadno
    spočítat horní odhad na pravděpodobnost, že padne více než $3n/4$
    hlav.

    \[
        P(X \geq 3n/4) \leq \frac{n/2}{3n/4} = \frac{2}{3}
    \]
\end{example}

\begin{theorem}
    Pokud známe pouze $E(X)$ a fakt, že $X$ nenabývá záporných hodnot,
    potom Markovova nerovnost je těsná.
\end{theorem}

\begin{proof}
    Nechť $X = k$ s pravděpodobností $1$
    a $X = 0$ s~pravděpodobností $0$.
    Potom $P(X \geq k) = \frac{E(X)}{k}$.
\end{proof}

Pro aplikaci Markovovy nerovnosti stačila znalost prvního momentu
($\Expect(X)$). Se znalostí druhého momentu ($\Expect(X^2)$) můžeme
odvodit podstatně silnější nerovnost.

\begin{theorem}[Čebyševova nerovnost]
    Pokud $a \geq 0$, pak
    \[
        P(\lvert X - \Expect(X) \rvert \geq a) \leq \frac{\Var(X)}{a^2}
    \]
\end{theorem}

\begin{proof}
    Nejprve pozoroujeme, že
    \[
        P(\lvert X - \Expect(X) \rvert \geq a)
        = P((X - E(X))^2 \geq a^2)
\]
    potom pouze použijeme Markovovu nerovnost
    \[
        P((X - \Expect(X))^2 \geq a^2)
        \leq \frac{\Expect((X - \Expect(X))^2)}{a^2}
        = \frac{\Var(X)}{a^2}
        \qedhere
    \]
\end{proof}

\begin{example}
    Analyzujeme znovu příklad s~pravděpodobností, že padne více než
    $3n/4$ hlav z~$n$ hodů mincí. Nejprve ověříme, že
    \[
        \Expect((X_i)^2) = \Expect(X_i) = 1/2
    \]\[
        Var(X_i) = \Expect((X_i)^2) - \Expect(X_i)^2 = 1/4
    \]
    Nyní s~použítím linearity sčítání nekovariantních náhodných
    proměnných
    \[
        \Var(X) = \Var(\sum X_i) = \sum \Var(X_i) = n/4
    \]
    Aplikujeme Čebyševovu nerovnost
    \[
        P(X \geq 3n/4 ) \leq P(\lvert X - \Expect(X) \rvert \geq n/4)
                     \leq \frac{\Var(X)}{(n/4)^2}
                     = \frac{n/4}{(n/4)^2}
                    = \frac{4}{n}
    \]

    Protože nerovnost omezuje pravděpodobnost ze dvou stran (ze symetrie
    dané absolutní hodnotou), získáváme ještě přesnější odhad $2/n$.
\end{example}

\subsection{Chernoffovy odhady}

% http://math.mit.edu/~goemans/18310S15/chernoff-notes.pdf

Se znalostí všech momentů náhodné proměnné můžeme odvodit ještě silnější
odhady pravděpodobností.

\begin{definition}
    Funkci $M_X(s) = \Expect(e^{s X})$ nazveme {\em generátor momentů}.
\end{definition}

Důvodem pro název je Taylorův rozvoj
$M_X(s)
= \Expect( 1 + sX + \frac{1}{2!}s^2 X^2 + \frac{1}{3!}s^3 X^3 + \ldots )
= \sum_{i = 0}^{\infty} \frac{1}{i!} s^i \Expect (X^i)$
a z něj vyplývající tvrzení (za předpokladu, že $\Expect$ a derivace
pro dané $X$ komutují) $\Expect(X^n) = M^{(n)}_X(0)$, kde $(n)$ značí $n$-tou
derivaci.

\begin{theorem}
    Pokud $X, Y$ jsou nezávislé, pak
    \[
        M_{X+Y}(t) = M_X(t) M_Y(t)
    \]
\end{theorem}

Důkaz je snadný z~definice.

\begin{definition}[Chernoffove odhady]
Obecně Chernoffovy odhady jsou získány aplikací Markovovy nerovnosti na
$e^{tX}$ pro vhodně zvolené $t$. Z Markovovy nerovnosti pro každé $t >
0$ platí
\[
    P(X \geq a) = P(e^{tX} \geq e^{ta}) \leq \frac{\Expect(e^{tX})}{e^{ta}}
\]
zejména tedy
\[
    P(X \geq a) \leq \min_{t > 0} \frac{\Expect(e^{tX})}{e^{ta}}
\]
Stejně tak platí opačná nerovnost pro $t < 0$.
\end{definition}

%Odvodíme Chernoffův odhad pro součet Poissonových pokusů\footnote{To je téměř
%jako součet Bernoulliho pokusů, jen každý Poissonův pokus může mít jinou
%pravděpodobnost, zatímco od Bernoulliho očekáváme stejnou ve všech
%případech.}.

Jeden z~Chernoffových odhadů je následující.

\begin{theorem}
    Nechť $X = \sum X_i$, kde $X_i$ je Poissonovo s~$p_i$,
    všechny $X_i$ jsou nezávislé a nechť $\delta \in (0,1)$.
    \[
        P(X \geq (1 + \delta) \Expect(X)) \geq e^{- \Expect(X) \delta^2 / 3}
    \]
\end{theorem}

\begin{proof}[Náčrt důkazu]
% Slidy IV111
% Probability and Computing (Mitzenmacher, Upfal)
% https://www.cs.cmu.edu/afs/cs/academic/class/15859-f04/www/scribes/lec9.pdf
% http://math.mit.edu/~goemans/18310S15/chernoff-notes.pdf

Začneme výpočtem $M_X(t) = \Expect(e^{tX})$, na což
ovšem potřebujeme $M_{X_i}(t)$.
\begin{align*}
    M_{X_i}(t) &= \Expect(e^{t X_i}) \\
    &= p_i e^t + (1 - p_i) \\
    &= 1 + p_i (e^t - 1) \\
    &\leq e^{p_i(e^t - 1)}
\end{align*}
Opět jsme použili $1 + x \leq e^x$. V následujícím použijeme dříve
uvedenou větu o součinu generátorů momentů.
\begin{align*}
    M_X(t) &= \prod M_{X_i}(t) \\
    &\leq \prod e^{p_i(e^t - 1)} \\
    &= \exp{\sum p_i(e^t - 1)} \\
    &= e^{(e^t-1)\Expect(X)}
\end{align*}
Máme tedy
\[
    P(X \geq a) \leq \frac{e^{(e^t-1)\Expect(X)}}{e^{ta}}
\]
substitucí $a = (1+\delta)\Expect(X)$, $t = \ln(1 + \delta)$,
\[
    P(X \geq (1 + \delta) \Expect(X)) \leq
    \frac{e^{\delta \Expect(X)}}{(1+\delta)^{(1+\delta)}}
\]
Že je výraz vpravo shora ohraničen $e^{- \Expect(X) \delta^2 / 3}$
necháváme k~uvěření.
\end{proof}

\begin{example}
    Opět uvážíme počet hlav spatřený během $n$ hodů mincí.
    \[
        P\left(X \geq \frac{3n}{4}\right)
        \leq 2 \exp \left\{ - \frac{1}{2} \frac{n}{2} \left(\frac{1}{2}\right)^2 \right\}
        = e^{-n/24}
    \]
    Získali jsme tedy exponenciální ohraničení pravděpodobnosti, že
    počet spatřených hlav je více než $3n/4$, zatímco Markovova
    nerovnost nám dala jen konstantní a Čebyševova lineární.
\end{example}

\subsection{Náhodné procesy}

\begin{definition}
    Nechť $S$ je spočetná množina a pro každé $u,v \in S$
    existuje $0 \leq p_{u,v} \leq 1$.
    Posloupnost náhodných proměnných $X_0, X_1, X_2, \ldots$
    s~{\em Markovovou vlastností}
    $P(X_n = a \mid X_{t-1} = b, \ldots, X_{0} = s)
    = P(X_n = a \mid X_{t-1} = b)
    = p_{b,a}$
    nazýváme {\em diskrétní, časově homogenní, Markovovský proces}.
\end{definition}

Vybereme-li počáteční stav $X_0$, můžeme na Markov chain nahlížet jako na
graf nebo automat. (Tady by se hodil příklad, ale spíš je to zřejmé.)

Charakterizujeme stavy Markovovských řetězců.

\begin{definition}
    Stav $i$ Markovovského řetězce nazveme {\em absorbující}
    právě tehdy, když $p_{i,i} = 1$.
\end{definition}

\begin{definition}
    Stav $i$ Markovoského řetězce nazveme {\em rekurentní}
    právě tehdy, když pravděpodobnost, že se ze stavů $i$ proces do něj
    opět vrátí je $1$.
\end{definition}

\begin{definition}
    Stav $i$ Markovoského řetězce nazveme {\em průchozí}
    právě tehdy, když pravděpodobnost, že se ze stavů $i$ proces do něj
    opět vrátí je menší než $1$ (tedy že se nevrátí je nenulová).
\end{definition}

(Tady by mohl být třeba obrázek, ale zase je to snad jasné.)

Nyní se podíváme na pravděpodobnosti a časy dosažení stavů
v~Markovovských řetězcích.

% http://www.statslab.cam.ac.uk/~james/Markov/s13.pdf

\begin{definition}
    Nechť $(X_n)_{n \in \mathbb{N}}$ je Markovovský řetězec.
    {\em Čas dosažení} stavů z~množiny $A \subseteq S$ je
    náhodná proměnná $H^A : S \to \mathbb{N}_{\infty}$,
    definovaná
    \[
        H^A(\omega) = \inf \{ n \geq 0 \mid X_n(\omega) \in A \}
    \]
    kde $\infty$ je infimum prázdné množiny.
\end{definition}

\begin{definition}
    Ve stavu $i$ je
    {\em pravděpodobnost dosažení} stavu z~množiny $A$
    \[
        h^A_i = P(H^A < \infty \mid X_0 = i)
    \]
    {\em Střední doba dosažení} stavu z~$A$ je
    \[
        k^A_i = \Expect(H^A \mid X_0 = i)
        = \sum_{n < \infty} n \cdot P(H^A = n) + \infty \cdot P(H^A = \infty)
    \]
\end{definition}

\begin{theorem}
    Vektor pravděpodobností dosažení $(h^A_i)_{i \in S}$
    je minimální nezáporné řešení systému rovnic
    \begin{align*}
        h^A_i &= 1 &\text{ pokud } i \in A, \\
        h^A_i &= \sum_j p_{i,j} h^A_j &\text{ pokud } i \not \in A.
    \end{align*}
\end{theorem}

\begin{example}
\end{example}

\begin{theorem}
    Vektor střední doby dosažení $(k^A_i)_{i \in S}$
    je minimální nezáporné řešení systému rovnic
    \begin{align*}
        k^A_i &= 0 &\text{ pokud } i \in A, \\
        k^A_i &= 1 + \sum_{j \not \in A} p_{i,j} k^A_j &\text{ pokud } i \not \in A.
    \end{align*}
\end{theorem}

\begin{example}
\end{example}

Jiné náhodné procesy než Markovovské řetězce jsou například martingales.

\begin{definition}
    Posloupnost náhodných proměnných $(X_i)_{i \in \mathbb{N}}$ nazveme
    {\em martingale}, pokud pro každé $n$
    \begin{itemize}
        \item $\Expect(\lvert X_n \rvert) = 0$,
        \item $\Expect(X_{n+1} \mid X_1, \ldots, X_n) = X_n$,
            tedy očekávaná příští hodnota podmíněná předchozími
            hodnotami je stejná jako současná hodnota.
    \end{itemize}
\end{definition}

Příklady: náhodná procházka po mřízce; stav konta gamblera, který hraje
fair hry.

\subsection{Entropie a informace}

Hledáme funkci pro popis entropie.
Snažíme se vyjádřit průměrné množství informací, které můžeme extrahovat
z~náhodného zdroje. Co bychom po takové funkci chtěli? Měla by
\begin{itemize}
    \item být spojitá,
    \item nabývat maximální hodnoty pokud pravděpodobnost všech hodnot
z~náhodného zdroje je stejná
    \item a navíc pokud je výskyt dvou části informace
nezávislý, potom entropie jejich současného výskytu by měla být rovna
součtu jejich entropií.
\end{itemize}

Analyticky lze dokázat, že následující definice
zachycuje právě takovou funkci.


% https://math.stackexchange.com/questions/331103/intuitive-explanation-of-entropy

\begin{definition}
    \[ H(X) = - \sum P(X = x) \log_b P(X = x) \]
\end{definition}

Intuitivně entropie říká, kolik nové informace můžeme ze zdroje dostat.
Například náhodný šum má velkou entropii, zatímco konstantní signál
nulovou.

\begin{example}
    Mějme minci $X$ se stranami $0,1$ a $P(X = 0) = 1/2$, $P(X = 1) =
    1/2$.
    $H(X) = - 1/2 (-1) - 1/2 (-1) = 1$.
\end{example}

Entropie náhodné proměnné dává dolní odhad na množství dat, které jsou
potřeba k~popisu jejich hodnot. Toho se hodí využít ke konstrukcí kódů.

\begin{definition}
    {\em Kód} $C$ pro náhodnou proměnnou $X$ je zobrazení
    $C : \Im(X) \to D^*$, kde $D$ je abeceda s~$\lvert D \rvert = d$.
    $l_C(x)$ značí délku kódového slova $C(x)$.
    Dále definujeme {\em očekávanou délku} kódu $C$.
    \[
        L_C(X) = \sum_{x \in \Im(X)} P(X = x) l_C(X) = \Expect(l_C(X))
    \]
\end{definition}

Huffmanovo kódování produkuje kódy, jejichž očekávaná délka je velice
blízko entropii zdroje dat.

\begin{example}
    Pro kód s~daným rozdělením slov sestrojíme Huffmanův kód a porovnáme
    jeho očekávanou délku s~entropií daného rozdělení.

    Mějme náhodnou proměnnou $X$ s~pravděpodobností jednotlivých hodnot
    $a,b,c,d,e$ po řadě
    $0.10,0.15,0.30,0.16,0.29$. Sestrojíme binární kód tak, že
    tvoříme odspodu strom spojením dvojic vrcholů, které dosud nemají
    společného rodiče a zároveň je součet jejich hodnot nejmenší
    z~takových dvojic. Do nově vzniklého vrcholu přiřadíme součet hodnot
    jeho potomků.

    Jakmile máme kořenový strom (nelze vybrat dva vrcholy, které nemají
    společného rodiče), přiřadíme hranám čísla 0,1. Cesta od kořene ke
    slovu potom zadává kód slova.

    Můžeme například přiřadit $a \mapsto 010, b \mapsto 011, c\mapsto
    11, d \mapsto 00, e \mapsto 10$.
    Což dává $L(C) = 0.30+0.45+0.60+0.32+0.58 = 2.25$.
    Naproti tomu $H(X) = 0.332+0.411+0.521+0.423+0.518 = 2.205$.
\end{example}

Poznamenejme, že Huffmanovy kódy jsou prefixové a tedy umožňují snadné
dekódování. V~konstrukci může být potřeba pro $n$-ární, $n > 2$, přidat
několik \uv{umělých} slov, aby kód vyšel s~optimální délkou.
