\section{Složitost}

\subsection{Úvod}

\subsubsection*{Znenie otázky}
Výpočetní, komunikační a Kolmogorovova složitost. Časové a paměťové 
složitosti a jejich vlastnosti; Třídy složitosti pro deterministické, 
nedeterministické a náhodnostní výpočty a vztahy mezi nimi; 
úplné, zejména NP-úplné, problémy; důkazy NP-úplnosti; 
P-NP problém a jeho implikace; aproximační řešení NP-problémů; 
aproximační třídy složitosti; interaktivní dokazovací systémy a 
zero-knowledge dokazovací systémy; komunikační složitost a její aplikace; 
metody dokazování dolních odhadů komunikační složitosti; 
třídy komunikační složitosti, základní pojmy a vlastnosti Kolmogorovovy složitosti. 

\subsubsection*{Pojmy}
$\mathcal{O}$-notácia, časová/priestorová zložitosť, determinizmus/nedeterminizmus,
linear speedup, vzťahy; uzavretosť tried na doplnky, nek. hierarchia tried, 
hierarchia (ostré inklúzie), randomizované triedy; redukcia (p, T, log), 
ťažké/úplné problémy, P/NP/PSPACE-úplné problémy; vzťah P-NP a jeho
implikácie; aproximatívne riešenia (MAX-CUT, MAX-SAT); interaktívne protokoly,
dIP[k], dIP, IP, zero-knowledge dôkazy, GI, GNI, bit commitment/opening phase;
fingerprinting, zníženie počtu odoslaných bitov; Kolmogorovova zložitosť, vlastnosti

\subsection{Časové a paměťové složitosti}

\begin{definition}
    Nechť $f, g : \mathbb{N} \to \mathbb{N}$. Řekneme, že
    $f \in \mathcal{O}(g)$, pokud
    \[
        \exists c, n_0 \; \forall n, n > n_0
        \text{ platí }
        f(n) \leq c \cdot g(n)
    \]
\end{definition}

\begin{definition}
Řekneme že Turingův stroj vyžaduje čas $T(n)$, $T : \mathbb{N} \to \mathbb{N}$,
pokud pro každý vstup $w$ provede nejvýše $T(\lvert w \rvert)$ kroků.

Řekneme že Turingův stroj vyžaduje prostor $S(n)$, $S : \mathbb{N} \to \mathbb{N}$,
pokud pro každý vstup $w$ použije nejvýše $S(\lvert w \rvert)$ míst
pásky.
\end{definition}

\begin{claim}
    Jestliže funkce $f : \{0, 1\}^* \to \{0, 1\}$ je funkce vyčíslitelná
    TS $M$ s~abecedou $\Gamma$ v~čase $T(n)$, potom je vyčíslitelná i
    TS $M'$ s abecedou $\{0,1\}$ v~čase $4 \log \lvert \Gamma \rvert
    \cdot T(n)$.
\end{claim}

Jednoduše každý prvek $\Gamma$ kódujeme pomocí $\log \lvert \Gamma
\rvert$ bitů.

\begin{claim}
    Jestliže funkce $f : \{0, 1\}^* \to \{0, 1\}$ je funkce vyčíslitelná
    TS $M$ s~$k$ páskami (a páskou pro vstup a páskou pro výstup) v~čase $T(n)$,
    potom je vyčíslitelná i TS $M'$ s~jedinou pracovní páskou (a páskou
    pro vstup a páskou pro výstup) v~čase
    $5 k \cdot T(n)^2$.
\end{claim}

Turingův stroj $M'$ kóduje obsah prvního políčka $k$ pásek pomocí pozic
$1, k+1, 2k+1,\ldots$, obsah druhého políčka pomocí $2, k+2,
2k+2,\ldots,$ a tak dále. Ke každému symbolu $a$ nabíc máme $\hat a$,
který značí, že na této pozici je hlava odpovídající pásky.
Každý krok výpočtu nejprve projde celou pásku pro nalezení pozic hlav
a potom provede změnu pásek dle definice $M$. Každý takový krok má tedy
složitost $5kT(n)$ (projítí celé pásky $M'$) a je jich $T(n)$ (dle složitosti $M$).

Definujeme základní třídy rozhodovacích problémů v~závistlosti
na potřebném čase pro jejich řešení. Nejprve však uvedeme větu, která
vysvětlí důvod multiplikativní konstanty u času vyžadovaného v~definici.

\subsection{Triedy zložitosti pre deterministické a nedeterministické výpočty}

Zavedeme některé užitečné třídy složitosti.

\begin{itemize}
    \item $\P = \bigcup DTIME(n^k)$
    \item $\NP = \bigcup NTIME(n^k)$
    \item $\PSPACE = \bigcup DSPACE(n^k)$
\end{itemize}
Platí například
\begin{itemize}
	\item $DTIME(f(n)) \subseteq NTIME(f(n))$, $DSPACE(f(n)) \subseteq NSPACE(f(n))$
    \item $\PSPACE = \NPSPACE$ (ze Savitchovy věty),
    \item jinak nevíme nic moc, ani třeba zda $\NP \cap \coNP$ (kde je
        například grafový izomorfismus) je $\P$.
	\item $DLOG \subseteq NLOG \subseteq \P \subseteq \NP \subseteq \PSPACE \subseteq EXPTIME \subseteq NEXPTIME \subseteq EXPSPACE$
\end{itemize}

\begin{theorem}[Lineární zrychlení]
    Nechť $M$ je TS se složitostí $T(n)$ a $c > 0$.
    Potom existuje TS $M'$, který řeší stejný problém
    se složitostí $1/2 T(n) + n + 2$.
\end{theorem}

Pro důkaz se zkonstruuje stroj se stavy reprezentujícími trojice stavů
původního automatu. Pozice pásky $i$ nového automatu reprezentuje stavy
$2i-1, 2i, 2i+1$, tedy trojice se překrývají. Neaktuální stavy ve
vedlejších trojicích je třeba při přechodu aktualizovat.

Podobně existuje věta o kompresi pásky.

\begin{itemize}
    \item $DTIME(T(n)) = \{ L(M) \mid \text{$M$ je det. TS vyžadující
        čas $c \cdot T(n)$}, c > 0 \}$
    \item $NTIME(T(n)) = \{ L(M) \mid \text{$M$ je nedet. TS vyžadující
    čas } c \cdot T(n), c > 0 \}$
\end{itemize}

Platí například, že pro každý nedeterministický stroj počítající v~čase
$T(n)$ existuje deterministický stroj počítající v~čase
$2^{\mathcal{O}(T(n))}$. A to protože nedeterministický stroj lze
simulovat deterministickým tak, že prohledává strom možný konfigurací do
šířky.

Pro složitostní třídy uvažujeme TS s~vyhrazenou read-only vstupní páskou
a write-only výstupní páskou, které do použitého prostoru nepočítáme.

\begin{itemize}
    \item $DSPACE(T(n)) = \{ L(M) \mid \text{$M$ je det. TS vyžad.
        prostor } c \cdot T(n), c > 0 \}$
    \item $NSPACE(T(n)) = \{ L(M) \mid \text{$M$ je nedet. TS vyžadující
        prostor nejvýše}$ $c \cdot T(n) \text{ v~každé větvi}, c > 0 \}$
\end{itemize}

\begin{theorem}[Savitch]
    Nechť $S : \mathbb{N} \to \mathbb{N}$, $S(n) \geq \log n$,
    potom $NSPACE(S(n)) \subseteq DSPACE(S(n)^2)$.
\end{theorem}

Savitch navrhl algoritmus \verb|k_edge_path|,
který zjišťuje, zda existuje cesta z~$s$ do $t$ délky~nejvýše $k$, a to
tak, že v~každém kroku rekurzivně zkusí, jestli existuje cesta z~$s$ do
nějakého (tj. zkusí všechny) $u$ délky $k/2$ a z~$u$ do $t$ délky $k/2$.
Algoritmus použije nejvýše $\log n$ úrovní a v~každé
$\mathcal{O}(\log n)$ bitů.
Výpočet stroje pro problém v~$NSPACE$ můžeme interpretovat jako graf
velikosti $2^{S(n)}$, počáteční stav nazvat $s$, zkusit koncové stavy
$t$ a za $k$ dosadit $n$. Celková složitost je tedy
$\mathcal{O}((\log 2^{S(n)})^2) = \mathcal{O}((S(n))^2)$.

\subsection{Triedy zložitosti pre náhodnostné výpočty}

(Nie som si istý, či je potrebné vedieť aj orákulové Turingove stroje
a polynomiálnu hierarchiu, ale keďže nižšie je spomenutá, aspoň ich spomeniem.)

Pre definovanie polynomiálnej hierarchie je potrebné zhrnúť princíp
orákulových Turingových strojov. Uveďme na príklade - jazyk je v~triede
$\P^{\NP}$ pokiaľ je rozhodovaný orákulovým Turingovým strojom,
ktorý beží v~deterministickom polynomiálnom čase, avšak môže sa počas
výpočtu neobmedzene veľa krát (ale iba polynomiálne veľa krát, inak by
už nebežal v~polynomiálnom čase) \uv{opýtať orákula}, ktoré mu 
rozhodne ľubovoľný problém v~triede $\NP$.

% TODO: toto je podľa mňa možno blbosť, či?
%Poznamenajme, že táto trieda nemá moc význam, keďže deterministický polynomiálny
%stroj nedokáže rozhodnúť nič navyše oproti tomu, čo by dokázalo
%orákulum rozhodnúť samo, tj. $\P^{\NP} = \NP$.

\begin{definition}[Polynomiálna hierarchia]
	Definujeme systém tried $\Sigma_i^\P$ a $\Pi_i^\P$ ako
	\begin{align*}
		\Sigma_0^\P = \P \\
		\Sigma_{i+1}^{\P} = \NP^{\Sigma_i^{\P}} \\
		\Pi_i^{\P} = co-\Sigma_i^{\P}
	\end{align*}
	Triedu $PH$ definujeme ako $PH=\bigcup_{i=0}^{\infty} \Sigma_i^\P \cup \Pi_i^\P$.
\end{definition}

Povolíme-li k deterministickým TS navíc pásku s~náhodnými bity získáváme pravděpodobnostní
Turingovy stroje (PTS) a třídy složitosti.

\begin{definition}
	Trieda $\RP$ obsahuje problémy rozhodnutelné v~polynomiálním čase 
	PTS s~jednostrannou chybou, tj.
	\begin{itemize}
		\item pokiaľ $x \in A$, $P(\mathcal{M}(x) \text{ akceptuje}) \geq 3/4$,
		\item pokiaľ $x \not\in A$, $P(\mathcal{M}(x) \text{ akceptuje}) = 0$.
	\end{itemize}
\end{definition}

\begin{definition}
	Trieda $\BPP$ obsahuje problémy rozhodnutelné v~polynomiálním čase 
	PTS s~oboustrannou chybou, tj.
	\begin{itemize}
		\item pokiaľ $x \in A$, $P(\mathcal{M}(x) \text{ akceptuje}) \geq 3/4$,
		\item pokiaľ $x \not\in A$, $P(\mathcal{M}(x) \text{ akceptuje}) \leq 1/4$.
	\end{itemize}
\end{definition}

\begin{definition}
	Triedu $\ZPP$ definujeme ako $\RP \cap \coRP$.
\end{definition}

Platí například
\begin{itemize}
    \item $\ZPP = \RP \cap \coRP$,
	\item $\ZPP \subseteq \RP$, $\ZPP \subseteq \coRP$
	\item $\RP \subseteq \NP$, $\coRP \subseteq \coNP$
    \item $\P \subseteq \ZPP$,
    \item vztah $\BPP$ a $\NP$ není znám,
    \item $\BPP \subseteq \Sigma_2 \cap \Pi_2$ (důkaz není snadný).
\end{itemize}

\subsection{Úplné problémy složitostních tříd}

\begin{definition}
    Nechť $A, B$ jsou rozhodovací problémy.
    Řekneme, že $A$ se {\em polynomiálně redukuje} na $B$,
    $A \leq_p B$, pokud existuje v~polynomiálním čase vyčíslitelná
    funkce $f$ taková, že $x \in A \iff f(x) \in B$.
\end{definition}

Obdobně můžeme definovat relaci redukce v~logaritmickém {\bf prostoru}
$\leq_{\log}$. Je důležité si uvědomit, že $\L \subseteq \P$,
jelikož TS používající $\mathcal{O}(\log n)$ prostoru nemůže
použít více než $2^{\mathcal{O}(\log n)} = n^{\mathcal{O}(1)}$ času
aniž by opakoval již projdené konfigurace. Platí tedy
$(A,B) \in {\leq_{\log}} \implies (A,B) \in {\leq_p}$.

\begin{definition}
    Nechť $\mathcal{C}$ je třída rozhodovacích problémů
    a $\preceq$ relace redukce.
    Problém $A$ nazveme $\preceq$-{\em těžký} v~$\mathcal{C}$,
    pokud pro každý problém $B \in \mathcal{C}$
    platí $B \preceq A$.
    Problém $A$ nazveme $\preceq$-{\em úplný} v~$\mathcal{C}$, pokud je
    $\preceq$-těžký v~$\mathcal{C}$ a $A \in \mathcal{C}$.
\end{definition}

Každý problém v~$\P$ je triviálně úplný vzhledem k~polynomiální redukci.
Příkladem $\P$ úplného problému vzhledem k~logaritmické redukci je
{\em Circuit Value Problem}.

\begin{definition}
    Instance {\em Circuit Value Problem (CVP)}
    je zadána konečně mnoha přiřazení tvaru
    \begin{itemize}
        \item $P_i \coloneqq 0$,
        \item $P_i \coloneqq 1,$
        \item $P_i \coloneqq P_j \land P_k, \quad j, k < i$,
        \item $P_i \coloneqq P_j \lor P_k, \quad j, k < i$,
        \item $P_i, \coloneqq \neg P_j, \quad j < i$.
    \end{itemize}
    kde každé $P_i$ se vyskytuje na levé straně právě jednou. Díky
    podmínkám $j,k < i$ je zaručena acyklicita. Cílem je spočítat $P_n$,
    kde $n$ je nejvyšší index.
\end{definition}

\begin{theorem}
    CVP je $\leq_{\log}$-úplný v~$\P$.
\end{theorem}

\begin{proof} (Náčrt.)
    Příšlušnost $CVP$ do $\P$ se ověří snadno. Odpovídající TS projde
    $P_i$ v pořadí $i = 1,\ldots,n$ a vyhodnotí je.

    Pro důkaz $\leq_{\log}$-úplnosti v~$\P$ zakódujeme výpočet
    deterministického TS vyžadujícího polynomiální množství času na
    daném vstupu jako instanci $CVP$.
    Základní myšlenka je definovat v~obvodu booleovské proměnné
    $P^{a}_{ij}, a \in \Gamma$, která značí, že na pásce je na pozici
    $j$ symbol $a$ v~čase $i$,
    $Q^q_{ij}, q \in Q$, která značí, že stroj v~čase $i$ je ve stavu $q$
    a hlavu má na pozici $j$.
    Konkrétněji můžeme například definovat $P^{b}_{ij}$ tak,
    že už na pozici $j$ symbol byl v čase $i-1$ a hlava stroje se tam
    nenachází {\em nebo} (tj. disjunkce)
    jako disjunkci přes všechny přechody tak, že stroj byl
    v~odpovídajícím stavu a s~odpovídajícím stavem pásky tak, aby byl
    povolen přechod.
    Detaily jsou v~{\em Theory of Computation} (Kozen).
    Konstrukce používá logaritmický prostor pro zapsání indexů,
    v~ostatních místech stačí prostor konstantní.
\end{proof}

\begin{theorem}[Cook-Levin]
    $SAT$ je $\NP$-úplný.
\end{theorem}

K důkazu můžeme použít myšlenky důkaz pro $CVP$.
Získaný $CVP$ transformujeme na formuli tak, že zvolíme
$P_i \Leftrightarrow E_i$ pro každé $P_i \coloneqq E_i$
a vezmeme konjunkci všech takto získaných formulí. Získaná formule je
splnitelná právě tehdy když $M$ akceptuje.

Standardnější, i když podobný, způsob důkazu je zakódování popisu
výpočtu každého nedeterministického TS do formule prvořádové logiky.

% https://www.cs.cmu.edu/~ckingsf/bioinfo-lectures/npcomplete.pdf

Nyní ukážeme důkaz $\NP$-těžkosti pomocí redukce. Připomeňme, že problém
vrcholového pokrytí je najít množinu takovou, že každá hrana v~ní má
alespoň jeden vrchol. Problém nezávislé množiny je pro $k$ se ptá, zda
v~grafu existuje množina alespoň $k$ vrcholů nespojených hranou.

\begin{theorem}
    $S$ je nezávislá množina $\iff$ $V \setminus S$ je vrcholové
    pokrytí.
\end{theorem}

\begin{proof}
$\implies$ Předpokládejme, že $S$ je nezávislá množina
a $e = (u,v)$ hrana. Pouze jeden z~$u,v$ je v~$S$,
tedy pouze jeden z~$u,v$ je v~$V \setminus S$, tedy $V \setminus S$ je
vrcholové pokrytí.

$\impliedby$ Předpokládejme, že $V \setminus S$ je vrcholové pokrytí
a $u, v \in S$. Mezi $u,v$ není hrana (jinak by nebyl pokrytý
v~$V \setminus S$). Takže $S$ je nezávislá množina.
\end{proof}

Problém nezávislé množiny se redukuje na problém vrcholového pokrytí.
Podle věty se na existenci nezávislé množiny velikosti $k$
je potřeba zeptat na existenci vrcholového pokrytí
$V \setminus S$ velikosti $ \leq \lvert V \rvert - k$.

Problém vrcholového pokrytí se redukuje na problém nezávislé množiny.
Podle věty se na zjištění existence vrcholového pokrytí o $k$ vrcholech
je potřeba se zeptat na nezávislou množinu velikosti $n-k$.

O obou problémech nyní víme, že jsou stejně těžké, jelikož se na sebe
redukují. Stačí vědět o jednom (předpokládejme, že máme důkaz), že je
$\NP$-těžký a nyní automaticky víme, že i ten druhý je $\NP$-těžký.
Důkaz příslušnosti do $\NP$ je snadný, problémy jsou tedy $\NP$-úplné.

Príkladom $\PSPACE$-úplného problému je $TQBF$, čo je problém splniteľnosti
plne kvantifikovanej formule. Plne kvantifikovaná formula je formula,
v~ktorej je každá premenná kvantifikovaná buď existenčným alebo všeobecným
kvantifikátorom. Dôkaz $\PSPACE$-úplnosti nebudeme uvádzať.

\subsection{$\P-\NP$ problém}

\begin{itemize}
\item $\P \neq \NP$, důkaz dává dolní odhad na řešení SAT se složitostí
    $2^{\lvert\varphi\rvert}$.
    Zůstává teorie i praxe.

\item $\P \neq \NP$, ale existuje algoritmus pro SAT se složitostí
    $2^{\lvert\varphi\rvert^{0.001}}$.
    Teorie zůstává jak jsme předpokládali, praxe se
    dost mění (zejména kryptografie).

\item $\P = \NP$, ale nejlepší polynomiální algoritmus pro SAT má
    složitost $2^{1000000!} \cdot \lvert \varphi \rvert$.  Teorie se
    dost mění, praxe zůstává stejná.

\item $\P = \NP$, nejlepší algoritmus pro SAT má složitost
    $c \cdot \lvert \varphi \rvert$, kde $c$ je přiměřeně malá konstanta
    vzhledem k současným možnostem hardware.
    Mění se teorie i praxe.
\end{itemize}

Čo sa týka dôsledkov v~teórii, pokiaľ by platilo $\P = \NP$, tak by 
platilo aj $\P = \NP = \coNP$, alebo aj $\P = PH$, čo by teda znamenalo,
že celá polynomiálna hierarchia by bola jedinou triedou.

Takisto by všetky $\NP$-úplné problémy boli v~$\P$, čo by mohlo znamenať
schopnosť efektívne riešiť problémy, na ktorých stojí súčasná kryptografia.

\subsection{Složitost aproximací}

O algoritmech více v~sekci o řešení těkých problémů.
(Vzhledem k~tomu, že složitost aproximací se nikde neprobírala (kromě
občasné zmínky \uv{pro tento problém neexistuje aproximační algoritmus}
na IA101), spokojím se zde s krátkým přehledem složitostních tříd.)
Připomeňme, že níže uvedené jsou třídy optimizačních problémů.
% https://en.wikipedia.org/wiki/Optimization_problem

\begin{definition}
    Problém patří do třídy $\APX$ ({\em approximable}), pokud pro něj existuje
    aproximativní algoritmus s~polynomiální složitostí
    a konstantním aproximativním poměrem.
\end{definition}

\begin{definition}
    Problém patří do třídy $\PTAS$ ({\em polynomial time approximation
    scheme}), pokud pro něj existuje
    aproximativní algoritmus, který pro daný vstup a $\varepsilon > 0$
    vypočte $(1+\varepsilon)$-aproximativní řešení
    v~čase polynomiálním vzhledem k~velikosti vstupu.
\end{definition}

Zejména třeba $n^{1/\varepsilon}$ je povolená
složitost, přestože nárust složitosti je se zmenšujícím se $\varepsilon$
exponenciální.

\begin{definition}
    Problém patří do třídy $\FPTAS$ ({\em fully polynomial time approximation
    scheme}), pokud pro něj existuje
    aproximativní algoritmus, který pro daný vstup a $\varepsilon > 0$
    vypočte $(1+\varepsilon)$-aproximativní řešení
    v~čase polynomiálním vzhledem k~velikosti vstupu a k~$1/\varepsilon$.
\end{definition}

Platí $\FPTAS \subseteq \PTAS \subseteq \APX$.

\subsection{Interaktivní a zero-knowledge dokazovací systémy}

\begin{definition}
    Nechť $P, V$ ({\em Prover, Verifier}) jsou Turingovy stroje s~dvěmi
    společnými páskami: vstupní a komunikační. $P$ je s neomezenými
    zdroji a $V$ má pracovní pásku, polynomiální čas a přístup k
    náhodným bitům. $V$ začíná výpočet a může přes pracovní pásku
    polynomiálněkrát položit dotaz $P$. Na konci $V$ buď akceptuje
    nebo zamítá.

    Protokol $(P,V)$ nazveme {\em interaktivní dokazovací systém} pro
    $A$, pokud pro všechny $x$,
    \begin{itemize}
        \item pokud $x \in A$, pak platí $P_y((P,V)$ akceptuje $x) > 1/2$,
        \item pokud $x \not \in A$, pak pro všechny $P'$ platí $P_y((P',V)$
            akceptuje $x) < 1/2$,
    \end{itemize}
    kde pravděpodobnost je přes uniformní distribuci bitových
    posloupností $y$ (delších než kolik může verifier použít).

    Třídu problémů rozhodnutelných interaktivním dokazovacím systémem nazýváme
    $\IP$.
\end{definition}

\begin{theorem}
    $SAT \in \IP$.
\end{theorem}

\begin{proof}
    Prover pošle splňující přiřazení, pokud existuje. Verifier
    akceptuje právě tehdy, když je formule s~přiřazením splněna.

    Pokud je $\phi \in SAT$, pak $(P,V)$ akceptuje s~pravděpodobností 1.
    Pokud $\phi \not \in SAT$, pak žádný $P'$ nepřesvědčí Verifiera o
    splnitelnosti $\phi$, tedy akceptuje s~pravděpodobností 0.
\end{proof}

\begin{theorem}
    Grafový {\bf ne}izomorfismus je v~$\IP$.
\end{theorem}

Grafový izomorfismus je v~$\NP$, nevíme ale, zda je v $\coNP$.
Důkaz je popsán v {\em Theory of Computation} (Kozen).


Zero-knowledge dokazovací systémy jsou takové, ve kterých $P$ neprozradí
$V$ žádnou informaci kromě platnosti nějakého tvrzení. Příkladem je
problém Peggy, barvoslepého Viktora a dvou jablek zcela identických až
na barvu. Peggy chce přesvědčit Viktora, že jablka jsou různá, aniž by
mu prozradila, která je které.

Viktor ukáže Peggy jablko v~levé ruce a v~pravé ruce. Následně s~nimi
za zády zamíchá (pro náhodný bit 0 zůstanou jablka tam kde byly, pro
náhodný bit 1 jsou v~opačné ruce), ukáže Peggy jablko v~pravé ruce a
zeptá se Peggy, jestli je to jablko, které bylo v pravé ruce předtím.
Pokud jablka mají různé barvy, Peggy odpoví správně, pokud ne, má
pravděpodobnost správné odpovědi $1/2$.
Experiment se opakuje a po $k$ opakováních má Peggy šanci na uhádnutí
jen $1/2^k$. Viktor uzná rozdílnost jablek po dostatečně velkém
$k$, případně dříve odhalí, že Peggy lže.

Protokol skutečně splňuje očekávání od zero-knowledge dokazovacího
systému. Je (pravděpodobnostně) korektní, jelikož Viktor není (až na
malou pravděpodobnost) nikdy přesvědčen o nepravdě.
Je úplný, protože pokud jsou jablka rozdílné barvy je Viktor vždy
přesvědčen. A je zero-knowledge, protože Peggy nikdy nevyzradí žádnou
informaci o barvě.

\subsection{Komunikační složitost}

Uvažujeme model dvou stran s~neomezenou výpočetní kapacitou.
Každá strana má svůj $n$-bitový vstup a nezná vstup pro druhou stranu.
Dohromady se snaží spočítat oběma známou funkci
$f : \{0,1\}^n \times \{0,1\}^n \to \{0,1\}$
pomocí předem ustanoveného protokolu. Složitost protokolu měříme v~počtech
bitů, které v~nejhorším případě putují po komunikačním kanálu.

\begin{example}
Problém parity má konstantní složitost.
\end{example}

\begin{example}
Problém porovnání řetězců má nejvýše lineární složitost.
\end{example}

Komunikační složitost byla přímo zavedena pro zkoumání dolních odhadů
pro složitost výpočtu. Rozdělíme-li problém mezi několik paralelně
běžících procesorů, můžeme je rozdělit na poloviny a zkoumat, kolik bitů
potřebují přenést, což přímo dává odhad na výpočetní složitost.

Další aplikací je například důkaz, že TS s~$S(n) \leq n$ prostorovou složitostí
potřebuje alespoň $\Omega(n / S(n))$ času pro rozhodnutí jazyka
$\{ x\#x \mid x \in \{0,1\}^* \}$. Komunikační složitost byla také
použita k~dokazování dolních odhadů složitosti operací na datových
strukturách jako jsou haldy.

Nyní ukážeme dva způsoby jak dokázat, že problém porovnání řetězců má
nejméně lineární složitost.

První je metoda {\em fooling set}.
Předpokládejme, že existuje protokol se složitostí nejvýše $n-1$ bitů.
Tedy existuje pro něj nejvýše $2^{n-1}$ možných komunikací mezi oběma
stranami. Avšak možných páru pro porovnání rovnosti je $2^n$.
Podle Dirichletova principu musí existovat páry $(x,x), (x',x')$ pro
které komunikují strany stejně podle tohoto protokolu.
Což ale znamená, že odpoví špatně na $(x,x')$.

Pro funkci $f$ je fooling set $(S,b)$ množina vstupů
a binární hodnota taková, že pro všechny $(x,y) \in S \; f(x,y) = b$,
ale pro dva vstupy $(x,y) \neq (x',y') \in S$ platí
$f(x,y') \neq b$ nebo $f(x',y) \neq b$.
Platí (bez důkazu), že složitost výpočtu funkce je alespoň logaritmus velikosti
fooling setu.
V~případě našeho problému je fooling set $\{(x,x) \mid x \in \{0,1\}^n \}$.

% http://www.cs.cmu.edu/~venkatg/teaching/ITCS-spr2013/notes/lect-apr2-4.pdf

Druhá je {\em rank method}. Funkci $f$ můžeme zapsat do matice,
potom (bez důkazu) platí, že složitost výpočtu funkce je alespoň
logaritmus hodnosti této matice. V~našem příkladu je $2^n$ možných
vstupů, matice pro problém EQ je přitom matice identity $I_{2^n}$,
tedy podle tvrzení získáváme složitost alespoň $n$.

Rank method je silnější, dá se s~ní například dokázat, že složitost
skalárního součinu je alespoň lineární, zatímco fooling set metodou lze
pouze logaritmický odhad.

$\P^{\CC}$ je třída všech funkcí, které jsou vyčíslitelné
v~polynomiálním počtem vyměněných bitů. Stejným způsobem se definují
další třídy komunikační složitosti. Většinou platí inkluze známe ze
standardní složitosti, a to ze stejných důvodů. (Tohle tvrdí Complexity
Zoo. Žádné jednoduché, zajímavé tvrzení o složitostních třídach jsem
nenašel, možná je něco v nějakých přehledových článcích, ale kvůli jedné
podotázce, která se nikde neprobírá, je číst nebudu.)


\subsection{Kolmogorovská složitost}

Zafixujme nějaký Turingovsky úplný programovací jazyk.  Kolmogorovská
složitost $K(s)$ řetězce $s$ je délka nejkratšího programu, který vypíše
$s$.

\begin{theorem}
    Existuje $c$ takové, že pro všechny $s$ platí
    $K(s) \leq |s| + c$.
\end{theorem}

\begin{proof}
Důkazem je program s kódem \verb|print(|$s$\verb|)|, kde se do $c$
schová název funkce pro tisk a závorky.
\end{proof}

\begin{theorem}
    Pro každé $n \in \mathbb{N}$ existuje $s$ takový, že $K(s) > n$.
\end{theorem}

\begin{proof}
V opačném případě by všech nekonečně mnoho možných řetězců bylo možné
generovat (konečně mnoha) programy kratší než $n$ bitů (pro nějaké $n$).
\end{proof}

\begin{theorem}
    $K$ není totálně vyčíslitelná.
\end{theorem}

\begin{proof}
Předpokládejme, že existuje funkce $K$ v~daném programovacím jazyce,
která počítá Kolmogorovskou složitost
a je dlouhá $X$ bitů. Navíc předpokládejme, že interpret našeho jazyka
je $Y$ bitů dlouhý. Vytvoříme funkci, která zkouší postupně všechny řetězce $s$ délky
$1,2,\ldots$ a počítá $K(s)$ než nalezne $s$ takový, že $K(s) \geq
X+Y+100000$. Taková funkce nalezne řetězec, jehož kolomogorovská
složitost je větší než $X+Y+100000$, tedy řetězec, který nelze
vyprodukovat kratším programem než $X+Y+100000$-bitovým.

Poznamenejme, že argument skutečně funguje, jelikož naše funkce je jistě
kratší než 100000 bitů, a $X$, $Y$ jsou konstanty.
\end{proof}

