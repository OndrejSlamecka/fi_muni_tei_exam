\section{Automaty}

\subsection{Deterministické automaty}

\begin{definition}
    Deterministický konečný automat (DFA) je zadán
    pracovní abecedou $\Sigma$,
    množinou stavů $S$,
    počátečním stavem $s_0$,
    množinou koncových stavů $F$,
    a přechodovou funkcí $\delta : S \times \Sigma \to S$.
\end{definition}

Není-li přechodová funkce totální, snadno získáme ekvivalentní automat
s~totální prechodovou funkcí tak, že přidáme nový stav, kam pošleme
všechny dosud nedefinované přechody. Od teď pracujeme pouze s~úplnými
automaty.

\begin{definition}
    Pro daný $\mathcal{M} = (\Sigma, S, s_0, F, \delta)$
    induktivně definujeme přechodovou funkci
    $\delta^* : S \times \Sigma^* \to S$.
    Pro libovolný stav $s \in S$, písmeno $a \in \Sigma$ a
    slovo $aw \in \Sigma^*$ (tedy začínající písmenem $a$).
\begin{align*}
    \delta^*(s, \epsilon) &= s \\
    \delta^*(s, aw) &= \delta^*(\delta(s,a), w)
\end{align*}
\end{definition}

\begin{definition}[Jazyk rozpoznávaný automatem]
    Funkce $L$ přiřazuje automatu jazyk. Tedy
    $L(\mathcal{M}) = \{ w \mid \delta^*(s_0, w) \in F \}$
    pro $\mathcal{M} = (\Sigma, S, s_0, F, \delta)$.
    %Definujeme funkci $L : \mathfrak{M} \to \powerset{\Sigma^*}$.
\end{definition}

\subsubsection{Minimalizace DFA}

\begin{definition}
    Nechť $\mathfrak{M}$ jsou automaty rozpoznávající nějaký daný jazyk.
    Buď $n_\mathcal{M}$ počet stavů automatu $\mathcal{M}$.
    Automat $\mathcal{M} \in \mathfrak{M}$ nazveme {\em minimální},
    jestliže $\mathcal{M} \in \argmin_{M \in \mathfrak{M}} n_M$.
\end{definition}

Důvodem ne-minimality nějakého automaty jsou stavy nedosažitelné a nerozlišitelné.
Nedosažitelné stavy odstraníme jednoduchou variantou DFS. Zbývá
ztotožnit nerozlišitelné.

%\begin{definition}
    %Nechť $L$ je jazyk nad $\Sigma$ a $x,y \in \Sigma^*$.
    %Řetězec $z \in \Sigma^*$ nazveme {\em rozlišující rozšíření},
    %jestliže právě jeden z řetězců $xz, yz$ patří do $L$.

    %Definujeme $x \sim_L y$ právě tehdy, když neexistuje rolišující
    %rozšíření $x$ a $y$.
%\end{definition}

Hopcroftův algoritmus pro výpočet minimálního automatu rozdělí stavy
automatu do dvou tříd ekvivalence (dle (ne)příslušnosti do $F$) a pak
postupně relaci zjemňuje. V každém kroku je prodloužena délka řetězů po
jejichž projítí mají dané stavy skončit ve stejné tříde.  Nakonec
algoritmus vrátí faktorizaci daného automatu dle konečné relace --
ve tříde ekvivalence jsou ty stavy, ze kterých výpočet
skončí stejně pro libovolné slovo.

\begin{algorithm}
\caption{Minimalizace DFA bez nedosažitelných stavů}
\begin{algorithmic}[1]
\Function{MinDFA}{$\mathcal{M} = (\Sigma, S, s_0, F, \delta)$}
    \State ${\equiv} \gets \{(p,q) \mid p \in F \iff q \in F \}$
    \While{${\equiv}$ changes}
        \State ${\equiv} \gets
            \{ (p,q) \mid p \equiv q \land \forall a \in \Sigma :
            \delta(p,a) \equiv \delta(q,a) \}$
    \EndWhile
    \State \Return Factorization of $\mathcal{M}$ by ${\equiv}$
\EndFunction
\end{algorithmic}
\end{algorithm}

\begin{example}
\end{example}

\subsubsection{Rozhodovací problémy o DFA}
% http://www.cse.msu.edu/~torng/360Book/RegLang/Decision/

Problém ekvivalence (rozpoznávání stejného jazyka) dvou DFA je
rozhodnutelný minimalizací obou automatů a jejich porovnáním.

Kromě ekvivalence můžeme rozhodovat další problémy. Například zda
\linebreak $L(\mathcal{M}) = \emptyset$ nebo zda $L(\mathcal{M})$ je
konečný.

$L(\mathcal{M})$ je neprázdný pravě tehdy, když $\mathcal{M}$
akceptuje jedno ze slov délky nejvýše $\lvert S \rvert -1$.
Kdyby akceptoval pouze delší slova, musel by se ve výpočtu některý stav
opakovat, výpočet by šel o získaný cyklus zkrátit a získali bychom tak
kratší akceptované slovo (stejný princip jako pumping lemma).

$L(\mathcal{M})$ je konečný pravě tehdy, když $\mathcal{M}$
akceptuje jedno ze slov délky $n$ až $2 \lvert S \rvert - 1$.
Akceptuje-li takové slovo, potom ve výpočtu existuje cyklus, jehož
odpovídající část ve slovu můžeme libovolně pumpovat a získat tak
nekonečně mnoho delších slov.

\subsection{Nedeterministické automaty}

\begin{definition}
    Nedeterministický konečný automat (NFA) je zadán
    pracovní abecedou $\Sigma$,
    množinou stavů $S$,
    počátečním stavem $s_0$,
    množinou koncových stavů $F$,
    a přechodovou funkcí $\delta : S \times \Sigma \to \powerset{S}$.
\end{definition}

Narozdíl od NFA se může vyskytovat ve více stavech zároveň. Každé NFA
však lze snadno převést na ekvivalentní DFA. Řešení dříve uvažovaných
problémů na DFA se tak přímočaře převede na řešení stejných problémů na
NFA.

Determinizace DFA se provede tzv. {\em powerset construction}. Pro
každou množinu stavů, kterou může NFA pro nějaký vstup obývat je
sestrojen jediný stav nového DFA. Projdením celého NFA na způsob
grafového prohledávání je zaručeno, že DFA je úplné.

NFA bychom mohli definovat tak, že povolíme $\epsilon$-kroky. Před
determinizací by potom bylo potřeba spočítat $\epsilon$-uzávěr každého
stavu. Počítat $\epsilon$-uzávěr TODO

\subsection{Pravděpodobnostní automaty}

Pravdepodobnostné automaty sú rozšírením nedeterministických
automatov, kde je prechodová funkcia nahradená pravdepodobnostnou
distribúciou.

Jazyky rozhodované pravdepodobnostnými automatmi sa nazývajú stochastické
a sú nadmnožinou regulárnych jazykov.

% http://www.sciencedirect.com/science/article/pii/S0890540112001149

(Tohle se nikde neučí a vygooglit k tomu jde jen těžko něco. Zřejmě
existuje nějaký polynomiální algoritmus pro rozhodování ekvivalence.
Některé PA jsou ekvivalentní DFA, takže pro ně je jistě možná i
minimalizace.)

\subsection{Regulární jazyky a pumping lemma}

\begin{definition}
Nechť $\Sigma$ je abeceda. Jazyk $R$ je regulární právě tehdy, když
existuje DFA $\mathcal{A}$ takový, že $L(\mathcal{A}) = R$.
\end{definition}

Ekvivalentně lze regulární jazyky definovat pomocí regulárních výrazů a
regulárních gramatik.

Regulární jazyky jsou uzavřené na sjednocení (důkaz: spojením
dvou regexů pomocí \verb|+|), zřetězení (důkaz: spojením dvou
regexů za sebe), doplněk (důkaz: automat pro doplněk je stejný
jako pro původní jazyk, jen má obrácené akceptující stavy), průnik
(důkaz: platí $A \cap B = \overline{\overline{A} \cup \overline{B}}$ a
uzavřenost na doplněk a sjednocení jsme již dokázali).

\begin{theorem}[Pumping lemma]
Nechť $L$ je regulární jazyk nad $\Sigma$.
Potom
$\exists p \geq 1$ takové,
že $\forall w \in L, \lvert w \rvert \geq p$,
platí, že existují $x,y,z \in \Sigma^*$ tak, že $w = xyz$ a zároveň

\begin{itemize}
    \item $\lvert y \rvert \geq 1$ (cyklus je neprázdný),
    \item $\lvert xy \rvert < p$ (je v prvních $p$ znacích),
    \item $xy^nz \in L$ pro každé $n \geq 0$ (lze libovolně pumpovat).
\end{itemize}
\end{theorem}

\begin{example}
    Mějme jazyk $L = \{ a^n b^n \mid n \in \mathbb{N} \}$,
    pro spor předpokládejme, že je regulární.
    Mějme $p,x,y,z,n$ jako výše a zvolme $w = a^p b^p$.
    Z lemmatu víme, že $xy < p$, tedy $y$ je tvořeno nenulovým počtem
    písmen $a$.  Potom z lemmatu taky musí platit $xy^nz \in L$, to však
    zřejmě není pravda už pro $n = 2$.
\end{example}

\subsection{Zásobníkové automaty a bezkontextové jazyky}

\begin{definition}
    Zásobníkový automat (PDA) je zadán
    pracovní abecedou $\Sigma$,
    zásobníkovou abecedou $\Gamma$,
    množinou stavů $S$,
    počátečním stavem $s_0$,
    počátečním zásobníkovým symbolem,
    množinou koncových stavů $F$,
    přechodovou funkcí
    $\delta : S \times (\Sigma \cup \{ \varepsilon \}) \times \Gamma \to
    \finpowerset{S \times \Gamma^*}$
    a navíc informací o způsobu akceptování: buď koncovým stavem, nebo
    zásobníkem.
\end{definition}

Například lze sestrojit PDA pro jazyk
$\{ 0^n 1^n \mid n \in \mathbb{N} \}$.
PDA můžeme rozdělit na nedeterministické a deterministické. Existuje
bezkontextový jazyk, který nelze analyzovat DPDA.

\begin{definition}
    Jazyk je bezkontextový (CFL), pokud je generován bezkontextovou
    gramatikou (viz dále).
\end{definition}

Platí například, že jsou-li $L, P$ CFL, potom $L \cup P$ je CFL (důkaz:
jednoduše vytvoříme nové PDA, které nedeterministicky přejde do PDA pro
$L$ či $P$).

\begin{example}
    Jazyky
    $\{ a^m b^n c^n \mid m, n \in \mathbb{N} \}$,
    $\{ a^n b^m c^n \mid m, n \in \mathbb{N} \}$
    jsou bezkontextové, ale
    jejich průnik už není (ukáže se pomocí pumping lemma pro CFL).

    Z pozorování, že
    $A \cap B = \overline{\overline{A} \cup \overline{B}}$
    vyplývá, že CFL nejsou uzavřené ani na doplněk.
\end{example}

\pagebreak
\subsection{Gramatiky}

% https://en.wikipedia.org/wiki/Chomsky_hierarchy#The_hierarchy

\begin{longtable}[]{llll}
\toprule
Gramatika & Jazyk & Automat & Pravidla \tabularnewline
\midrule
\endhead
Typ-0 & Částečně rozhodnutelné & Turingův stroj & $\alpha \rightarrow \beta$ \tabularnewline
Typ-1 & Kontextové & Linérně ohr. TM & $\alpha A \beta \rightarrow \alpha \gamma \beta$ \tabularnewline
Typ-2 & CFL & (N)PDA &	$ A\rightarrow \gamma$ \tabularnewline
Typ-3 & Regulární & DFA & $A \rightarrow a$, $A \rightarrow aB$ \tabularnewline
\bottomrule
\end{longtable}

\vtop{\hbox{\strut Význam symbolů:}
\hbox{\strut \(a\) =
terminal}\hbox{\strut \(\alpha\) = terminal, non-terminal, or
empty}\hbox{\strut \(\beta\) = terminal, non-terminal, or
empty}\hbox{\strut \(\gamma\) = terminal or
non-terminal}\hbox{\strut \(A\) = non-terminal}\hbox{\strut \(B\) =
non-terminal}}

\bigskip

\begin{definition}
    Gramatika je v {\em Greibachové normální formě}, pokud
    každé pravidlo je tvaru
    $A \to a A_1 A_2 \ldots A_n$ nebo $S \to \varepsilon$.
\end{definition}

Každý top-down parser (zejména recursive descent parser)
zastaví nejvýše v hloubce rovné délce parsovaného řetězce. Prostorová
složitost parsování gramatiky v CFG je tedy lineární.

Navíc každou CFG lze převést do GNF, což dává důkaz, že každá CFG je
rozpoznatelná PDA.

\begin{definition}
    Gramatika je v Chomského normální formě, pokud každé pravidlo je
    tvaru $A \to BC$ nebo $A \to a$ nebo $S \to \varepsilon$.
\end{definition}

Parsovací strom takové gramatiky je binární. Převod do této formy se
použije v CYK algoritmu.


\subsection{Syntaktická analýza CFL}

Bezkontextové gramatiky můžeme analyzovat nedterministickým PDA
{\em shora dolů}, {\em zdola nahorů} nebo také algoritmem CYK na bázi
dynamického programování v~čase $\mathcal{O}(n^3 \lvert G \rvert)$.

Některé podmnožiny CFG bezkontextových gramatik lze analyzovat rychleji.
Zejména popíšeme LL(1) a LR(0) gramatiky, kterými se dají popsat běžné
programovací jazyky. Rychlost parsování je lineární v~součinu velikosti
vstupu a velikosti parsovací tabulky\footnote{TODO: Chtělo by to asi hlubší
zamyšlení, jestli to tak opravdu je a jak velká může být tabulka, ale
IMO je to zhruba velikost gramatiky na k-tou.}.

\subsubsection{Nedeterministická analýza}

Pro obecnou CFG můžeme sestrojit nedeterministické PDA. Rozlišujeme dva
základní způsoby parsování: shora dolů, zdola nahorů.

Metoda shora dolů je realizována jednostavovým automatem, který ve svém
zásobníku simuluje levé derivace gramatiky tak, aby platilo:
s~obsahem zásobníku $\alpha$ a zbytkem vstupu $w$ automat akceptuje
prázdným zásobníkem $\iff$ $\alpha \Rightarrow^* w$.

Rozlišujeme dvě akce automatu, které se opakují.
Je-li na vrcholu zásobníku neterminál
$X$, potom jej nahradíme pravou stranou nějakého pravidla $X \to
\ldots$. Je-li na vrcholu zásobníku terminál $x$, potom ověříme,
že ukazatel na čtené slovo ukazuje na znak $x$ a oba znaky umažeme.

\begin{example}

\end{example}

Metoda zdola nahorů použije dvoustavový rozšířený\footnote{Rozšířený
PDA může z~vrcholu zásobníku vybrat libovolný počet znaků.} automat, druhý
stav však slouží jen jako koncový. Automat práci začne
nad vstupní větou a skončí s~prázdným vstupem a s~kořenem gramatiky na
zásobníku.

Začne-li automat pracovat nad vstupem, který je větou z jazyka, pak
platí: $\alpha A y$ je obsah zásobníku zřetězený se zbytkem vstupu
$\iff$ $a A y$ je pravá větná forma. Rozlišujeme tři akce automatu,
které se opakují:
\begin{enumerate}
    \item $\delta(q, a, \epsilon) = \{ (q, a) \}, a \in \Sigma$,
    \item je-li $A \to \alpha$, pak $\delta(q, \varepsilon, \alpha)$
        obsahuje $(q, A)$,
    \item $\delta(q, \varepsilon, \bot S) = \{ (r, \varepsilon) \}$.
\end{enumerate}

\begin{example}
\end{example}


\subsubsection{CYK algoritmus}

CYK algoritmus je narozdíl od předchozích metod deterministický.

\begin{Verbatim}[fontsize=\small]
let the input be a string I consisting of n characters: a1 ... an.
let the grammar contain r nonterminal symbols R1 ... Rr, with start symbol R1.
let P[n,n,r] be an array of booleans. Initialize all elements of P to false.
for each s = 1 to n
  for each unit production Rv -> as
    set P[1,s,v] = true
for each l = 2 to n -- Length of span
  for each s = 1 to n-l+1 -- Start of span
    for each p = 1 to l-1 -- Partition of span
      for each production Ra -> Rb Rc
        if P[p,s,b] and P[l-p,s+p,c] then set P[l,s,a] = true
if P[n,1,1] is true then
  I is member of language
else
  I is not member of language
\end{Verbatim}

\subsubsection{LL(1) gramatiky}

% https://www.fi.muni.cz/usr/kretinsky/LL1aSLL.pdf

Nejprve zavedeme obecně LL(k) gramatiky, nakonec se ale zaměříme na
analýzu nejpoužívanější varianty, a to LL(1).

\begin{definition}
    Funkce $FIRST^G_k(\alpha)$ pro dané $\alpha \in (N \cup \Sigma)^+$
    vrátí všechny řetězce délky až $k$, které se dají z~$\alpha$
    odvodit.

    Funkce $FOLLOW^G_k(A)$ pro dané $A \in N$
    vrátí všechny řetězce délky až $k$, které mohou po terminálu $A$
    následovat.

    Binární operátor ${:}$ definujeme pro $k : w$ jako prvních $k$ znaků
    ze slova $w$ (nebo celé $w$ pro $w$ kratší než $k$).
\end{definition}

\begin{definition}
    Nechť $G = (N, \Sigma, P, S)$ je CFG, $k \geq 1$. $G$ nazveme LL(k)
    právě tehdy, když pro libovolné dvě nejlevější derivace
    $S \Rightarrow^* wA\alpha \Rightarrow w\beta\alpha \Rightarrow^* wx$,
    $S \Rightarrow^* wA\alpha \Rightarrow w\gamma\alpha \Rightarrow^* wy$,
    platí, že $k : x = k : y \implies \beta = \gamma$.
\end{definition}

\begin{theorem}
    Každá LL(k) gramatika je jednoznačná.
\end{theorem}

\begin{proof}
Pokud by nebyla jednoznačná, pak by existovala věta s~dvěmi odvozeními.
Potom by ale neplatila implikace v~definici.
\end{proof}

Věta: Levorekurzivní implikuje, že není ll(k) pro žádné k.

Gramatika je LL(k) právě když pro každé dvě větné formy $\beta \alpha$,
$\gamma \alpha$ a pravidla $A \to \beta, A \to \gamma$
platí $FIRST_k(\beta \alpha) \cap FIRST_k(\gamma \alpha) = \emptyset$.
Pro případ $k = 1$ však stačí snáze ověřitelná podmínka.

\begin{theorem}
    $G = (N, \Sigma, P, S)$ je LL(1) právě tehdy, když
    pro každé $A \in N$
    a každé dvě pravidla $A \to \beta, A \to \gamma$
    platí
    \[
        FIRST_1(\beta FOLLOW_1(A)) \cap FIRST_1(\gamma FOLLOW_1(A)) = \emptyset
    \]
\end{theorem}

Nyní ukážeme, jak sestrojit DPDA pro danou LL(1) gramatiku.
DPDA bude mít jeden stav, akceptuje prázdným zásobníkem
a začíná s~$S\$$ na zásobníku, kde $S$ je kořen
gramatiky a $\$$ se v gramatice nevyskytuje.

Definujme přechodovou funkci $M$.
Pro pravidlo $A \to \alpha$, potom pro každé $a \in FIRST_1(\alpha)$
definujeme $M(A, a) = \alpha$. Pokud $\varepsilon \in FIRST_1(\alpha)$,
potom pro každé $b \in FOLLOW_1(A)$ definujeme $M(A,b) = \alpha$.
Pro $a \in (\Sigma \cup \{ \$ \}$ definujeme $M(a,a) =$ {\em odstraň},
pro $x, a \in Sigma, x \neq a$ definujeme $M(x, a) =$ {\em zamítni}.

% https://www.cs.bgu.ac.il/~comp151/wiki.files/ps6.html#sec-2

\begin{example}
Mějme gramatiku $G$ s následujícími pravidly
\begin{verbatim}
S -> A
A -> aAb | c
\end{verbatim}
Přechodová tabulka má řádek pro neterminál $A$,
pod symbolem $a$ je pravidlo $A \to aAb$, pod $b$ není nic a pod $c$ je
$A \to c$.

\pagebreak

Mějme vstup \verb|aacbb$|, analýza probíhá takto:

\medskip
\centering
\begin{tabular}{|l|l|l|}
\hline
           & \textbf{Zbývající vstup} & \textbf{Zásobník}    \\ \hline
\textbf{1} & \verb|aacbb$|                  & \verb|A$|      \\ \hline
\textbf{2} & \verb|aacbb$|                  & \verb|aAb$|    \\ \hline
\textbf{3} & \verb|acbb$|                   & \verb|Ab$|     \\ \hline
\textbf{4} & \verb|acbb$|                   & \verb|aAbb$|   \\ \hline
\textbf{5} & \verb|cbb$|                    & \verb|Abb$|    \\ \hline
\textbf{6} & \verb|cbb$|                    & \verb|cbb$|    \\ \hline
\textbf{7} & \verb|bb$|                     & \verb|bb$|     \\ \hline
\textbf{8} & \verb|b$|                      & \verb|b$|      \\ \hline
\textbf{8} & \verb|$|                       & \verb|$|       \\ \hline
\textbf{9} & $\varepsilon$                              & $\varepsilon$          \\ \hline
\end{tabular}
\end{example}

\subsubsection{LR(0) gramatiky}

% https://is.muni.cz/auth/el/1433/podzim2015/IA006/um/LR0.pdf
% https://www.cs.bgu.ac.il/~comp151/wiki.files/ps6.html#sec-2
% https://stackoverflow.com/questions/7378337/what-is-the-difference-between-lr0-and-slr-parsing

LR gramatiky jsou naopak takové, které jsou rozpoznávané deterministickými PDA,
které postupují ve směru zdola nahoru. Zaměříme se na nejčastěji
používané LR(0) gramatiky.

Pro zjednodušení v~každé gramatice k~LR parsování nahradíme počáteční
neterminál $S$ novým $S'$ a přidáme pravidlo $S' \to S\$$. Za každý
vstup navíc zapíšeme ještě před začátkem analýzy $\$$.

\begin{definition}
    Nechť $G = (N, \Sigma, P, S)$ je CFG.
    {\em Položkou gramatiky} $G$ nazveme výraz tvaru
    $X \to \alpha \bullet \beta$, kde $X \to \alpha \beta \in P$.

    Pro každé $X \to \gamma \in P$ nazýváme $X \to \gamma \bullet$
    {\em úplnou položkou}.
\end{definition}

Nyní definujeme funkci {\em položkového uzávěru}
\[
    c(I) = I \cup
    \{ A \to \bullet \alpha \text{ pro každé } B \to \beta \bullet A \gamma \in I \},
\]
tedy funkci, která k~danému stavu parsování přiřadí všechny
\uv{ekvivalentní} stavy. Dále definujeme funkci (\uv{goto})
\[
g(I, a) = \{ B \to \alpha a \bullet \beta \text{ pro každé }
    B \to \alpha \bullet a \beta \in I \},
\]
která posune průběh analýzy o jeden znak (může být i neterminál).
S těmito dvěma funkcemi můžeme definovat LR(0) gramatiky.

\begin{definition}
    Gramatika $G$ je LR(0), pokud
    \begin{enumerate}
        \item pro libovolnou položku $\gamma$ se v $c(\gamma)$ vyskytuje
            nejvýše jedna úplná položka;
        \item jestliže se v~$c(\gamma)$ vyskytuje úplná položka, potom
            už v~$c(\gamma)$ není obsažena žádná položka, v~níž
            je bezprostředně vpravo od $\bullet$ terminální symbol.
    \end{enumerate}
\end{definition}

Nyní průchodem ve stylu DFS sestrojíme automat. Začneme s~počátečním
pravidlem a vytvoříme stav reprezentující $c(S' \to \bullet S\$)$.
Pro každý terminál či neterminál $a$ vyskytující se
za $\bullet$ teď vytvoříme nový stav $c(g(c(S' \to \bullet S\$), a))$
a vedeme hranu z~dosud zpracovávaného stavu.
Postup opakujeme, stav do kterého se přijde čtením $\$$ je koncový.

Pokud by gramatika nebyla LR(0), potom snadno ověříme v~zakresleném
automatu, že není splněna jedna z~části definice.

\begin{example}
    \href{https://www.cs.bgu.ac.il/~comp151/wiki.files/ps6.html\#sec-2}
{https://www.cs.bgu.ac.il/{\textasciitilde}comp151/wiki.files/ps6.html\#sec-2}
\end{example}


\subsection{Konečné automaty nad nekonečnými slovy}

% https://www7.in.tum.de/~esparza/autoskript.pdf

Pro automaty nad nekonečnými slovy používáme výraz $\omega$-automaty.
$\omega$-slovo je nekonečně dlouhé slovo.

$\omega$-automaty slouží jako konečná reprezentace množin nekonečných
slov. Nekonečnými slovy lze popsat například reálná čísla nebo průběh
programů.

Běžné operace na jazycích se provádí stejně na $\omega$-jazycích.
Drobným rozdílem je $\omega$ iterace, která je definována
$L^\omega = \{ (w_i)_{i \in \mathbb{N}} \mid w_i \in L \setminus \{ \varepsilon \} \}$,
tedy zejména
$\emptyset^\omega = \emptyset$, zatímco u Kleeneho iterace
$\emptyset^* = \{ \varepsilon \}$.

\begin{example}
    Intuitivním způsobem můžeme definovat
    $\omega$-regulární výrazy jako rozšíření regulárních výrazů.
    Např. $\emptyset$ je $\omega$-regulární, jelikož
    $L_\omega(\emptyset^\omega) = \emptyset$.
    Dále $(a+b)^\omega$ značí jazyk nekonečně dlouhých slov z~písmen
    $a,b$;
    $(a+b)^*b^\omega$ značí jazyk $\omega$-slov s konečně mnoha $a$;
    $(a+b)*ab)^\omega$ je jazyk slov obsahujících nekonečně mnoho $a$ i $b$.
\end{example}

\begin{definition}
    Nedeterministický $\omega$-automat je zadán
    pracovní abecedou $\Sigma$,
    množinou stavů $S$,
    počátečními stavy $S_0 \subseteq S$,
    akceptující podmínkou $S_{Acc} \subseteq S^\omega$
    a přechodovou funkcí $\delta : S \times \Sigma \to \powerset{S}$.
\end{definition}

Automat nazveme {\em deterministický}, pokud
$\lvert \delta(s,a) \rvert \leq 1$ pro každé $s,a$.

Posloupnost navštívených stavů při výpočtu na slově $w$ nazveme {\em běh
na $w$}.
$\omega$-automat akceptuje $w \in \Sigma^\omega$,
pokud běh na $w$ leží v~$S_{Acc}$.

$\omega$-automaty řadíme do několika kategorií podle volby akceptující
podmínky.

\begin{definition}
    Nechť $F \subseteq S$ je nějaká množina.
    $\omega$-automat nazveme {\em Büchiho},
    pokud pro každý běh $\rho \in S_{Acc}$ platí
    $\Inf{\rho} \cap F \neq \emptyset$.
\end{definition}

Büchiho automaty jsou tedy takové, které akceptují právě slova, které
navštíví některý z vybraných stavů nekonečněkrát.

\begin{example}
    Nechť jazyk $L$ nad $\{0,1\}$ obsahuje všechna $\omega$-slova,
    ve kterých se $1$ vyskytuje nekonečněkrát.
    $L$ je lze rozpoznat deterministickým Büchiho automatem.

\begin{center}
\begin{tikzpicture}[shorten >=1pt,node distance=2cm,on grid,auto]
   \node[state,initial] (q_0)  {$q_0$};
   \node[state,accepting](q_1) [right=of q_0] {$q_1$};
    \path[->]
    (q_0) edge [bend left]  node {1} (q_1)
          edge [loop above] node {0} ()
    (q_1) edge [loop above] node {1} ()
          edge [bend left]  node {0} (q_0);
\end{tikzpicture}
\end{center}
\end{example}

\begin{example}
    Nechť jazyk $L$ nad $\{0,1\}$ obsahuje všechna $\omega$-slova, ve
    kterých se $1$ vyskytuje pouze konečněkrát.
    $L$ je lze rozpoznat nedeterministickým Büchiho automatem.

\begin{center}
\begin{tikzpicture}[shorten >=1pt,node distance=2cm,on grid,auto]
   \node[state,initial] (q_0)  {$q_0$};
   \node[state,accepting](q_1) [right=of q_0] {$q_1$};
    \path[->]
    (q_0) edge  node {0} (q_1)
         edge [loop above] node {0} ()
         edge [loop below] node {1} ()
    (q_1) edge [loop above] node {0} ();
\end{tikzpicture}
\end{center}
\end{example}

\begin{claim}
    Deterministický Büchiho automat rozpoznávající jazyk
    $(0+1)^*0^\omega$ neexistuje.
\end{claim}

Intuitivně NBA \uv{uhodne} poslední výskyt $1$,
deterministický by na to potřeboval nekonečně mnoho stavů.

Můžeme uvažovat i jiné podmínky než Büchiho.

\begin{definition}
    Nechť $\Omega = \{(E_i, F_i)\}_{1 \leq i \leq n}$ pro nějaké
    $n \in \mathbb{N}$.
    $\omega$-automat nazveme {\em Rabinův},
    pokud pro každý běh $\rho \in S_{Acc}$
    existuje $i$ takové, že \linebreak
    $E_i \cap \Inf{\rho} = \emptyset$
    a
    $F_i \cap \Inf{\rho} \neq \emptyset$.
\end{definition}

Pokud dostaneme NBA s podmínkou $\{q_1, \ldots, q_k\}$,
snadno odvodíme NRA s podmínkou $\{(q_1, \emptyset, \ldots, (q_k, \emptyset))\}$.

\begin{example}
\end{example}

Nyní charakterizujeme jazyky rozpoznávané deterministickými $\omega$-automaty.

\begin{definition}
    Nechť $L$ je jazyk slov s~konečnou délkou.
    $\omega$-slovo $w$ je v limitě $L$ právě tehdy,
    když průnik všech prefixů $w$ s $L$ je nekonečná množina.
\end{definition}

Jinak řečeno pro každé $n$ lze v~$L$ najít slovo delší $n$ takové, že
je prefixem $w$.

\begin{theorem}
$\omega$-jazyk je rozpoznatelný deterministickým Büchiho automatem
právě tehdy, když je limitním jazykem nějakého regulární jazyka.
\end{theorem}

% Důkaz https://en.wikipedia.org/wiki/B%C3%BCchi_automaton

\subsection{Turingovy stroje}

\begin{definition}
    Turingův stroj (TM) je zadán
    páskovou abecedou $\Gamma$,
    symbolem prázdného políčka $b \in \Gamma$,
    vstupní abecedou $\Sigma \subseteq \Gamma \setminus \{b\}$,
    množinou stavů $S$,
    počátečním stavem $s_0$,
    množinou koncových stavů $F$,
    a přechodovou funkcí \linebreak
    $\delta : (S \setminus F) \times \Gamma
        \to S \times \Gamma \times \{L,R\}$.
\end{definition}

\begin{definition}[Univerzální TS]
Turingův stroj nazveme {\em univerzální},
který pokud na vstupu dostane dvojici kódu stroje a vstupu
$(\langle M \rangle, w)$, tak emuluje výpočet $M$ na $w$.
\end{definition}

Je třeba si dát pozor na technické problémy jako například různě velké
pracovní abecedy.

\begin{definition}[Rozhodnutelnost a vyčíslitelnost]
    Jazyk $L$ je {\em částečně rozhodnutelný}, pokud existuje Turingův
    stroj, který každý vstup $w$ akceptuje právě tehdy, když $w \in L$.

    Jazyk je {\em rozhodnutelný}, pokud je částečně rozhodnutelný
    strojem $M$ a navíc stroj $M$ zastaví na každém vstupu $w$.

    Funkce $f$ je {\em vyčíslitelná}, pokud existuje Turingův
    stroj $M$ takový, že pro $w$ z definičního oboru $f$ zastaví a na pásce má
    uloženo $f(w)$, pro ostatní $w$ cyklí.

    Funkce $f$ je {\em totálně vyčíslitelná}, pokud je vyčíslitelná a
    navíc totální.
\end{definition}


\begin{theorem}[Nerozhodnutelnost problému zastavení]
Mějme nějakou enumeraci vyčíslitelných Turingových strojů.
Následující funkce není vyčíslitelná.
\[
    f(i) =
    \begin{cases}
        1, & \text{jestliže program } i \text { zastaví na vstupu } i \\
        0, & \text{jestliže program } i \text { nezastaví na vstupu } i \\
    \end{cases}
\]
\end{theorem}

\begin{proof}
Dokážeme sporem. Nechť
\[
    g(i) =
    \begin{cases}
        \bot, & \text{jestliže } f(i) = 1 \\
        1, & \text{jestliže } f(i) = 0 \\
    \end{cases}
\]
Funkce $g$ je zřejmě částečně vyčíslitelná. Nechť $e$ je index stroje
počítajícího $g$ v dané enumeraci.

Pokud $g(e)$ je definováno, pak $f(e) = 0$ a tedy $g(e)$ nezastaví. Naopak
pokud $g(e)$ není definováno, pak $f(e) = 1$ a tedy $g(e)$ zastaví.
Dostáváme spor.
\end{proof}

Všimneme si, že v pomyslné tabulce programů a vstupů se funkce $g$ chová
na diagonále jinak, než každý program, což dává spor s její
vyčíslitelností.

Problém zastavení je však zřejmě částečně rozhodnutelný.
Naopak problém nezastavení není ani částečně rozhodnutelný (bez důkazu).

\subsection{RAM a PRAM modely}

% http://pages.cs.wisc.edu/~tvrdik/2/html/Section2.html

RAM je idealizovaný model běžného počítače. Paměť obsahuje libovolně
mnoho libovolně velkých buněk. Výpočetní jednotka nabízí
instrukce podobné běžným assemblerům, každá trvá jednotku času nehledě
na velikost operandů. Časová a prostorová složitost tak odpovídají počtu
provedených instrukcí a počtu použitých paměťových buněk.

PRAM je idealizovaný model SIMD (single instruction, multiple data) se
sdílenou pamětí. Obsahuje
neomezeně mnoho RAM procesorů $P_0, P_1,\ldots$
a neomezeně mnoho paměťových buněk $M[0], M[1],\ldots$

Procesory provádí synchroně operace {\em čti z buňky} $M$, proveď svůj
výpočet, {\em zapiš do buňky} $M$.
$P_0$ nejprve spočítá počet potřebných procesorů, teprve potom probíhá
výpočet. Výpočet končí se zastavením $P_0$. Časová složitost je čas
do zastavení $P_0$ (paralelismus!).

Rozlišujeme různé kategorie PRAM modelů dle řešení konfliktů při čtení a
zápisu. Viz
\href{http://pages.cs.wisc.edu/~tvrdik/2/html/Section2.html#Handling}{http://pages.cs.wisc.edu/{\textasciitilde}tvrdik/2/html/Section2.html\#Handling}\\
\href{https://en.wikipedia.org/wiki/Parallel_random-access_machine#Read/write_conflicts}{https://en.wikipedia.org/wiki/Parallel\_random-access\_machine\#Read/write\_conflicts}.

\subsection{Celulární automaty}

Například {\em Game of Life}, celulární automat na dvourozměrné, nekonečné
mřížce, kde každá buňka nabývá stavu {\em live} nebo {\em dead}.

\begin{enumerate}
    \item Any live cell with fewer than two live neighbours dies, as if caused by underpopulation.
    \item Any live cell with two or three live neighbours lives on to the next generation.
    \item Any live cell with more than three live neighbours dies, as if by overpopulation.
    \item Any dead cell with exactly three live neighbours becomes a live cell, as if by reproduction.
\end{enumerate}

Tato jednoduchá pravidla dávají snadno za vznik například tzv. {\em
gliderům}, objektům, které cestují po mřížce.
V tomto formalismu je možno zakódovat univerzální Turingův stroj.

Celulární automaty lze rozdělit do tříd dle stability a
chaotičnosti jejich vývoje. Již pro jednorozměrné
mřížky existují pravidla, která demonstrují jak chaotické chování
(pravidlo 30), tak částečně pravidelné a částečně chaotické (pravidlo
110).
