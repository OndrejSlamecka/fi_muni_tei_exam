\section{Grafy a grafové algoritmy}

\subsection{Úvod}
\subsubsection*{Znenie otázky}
Grafy a grafové algoritmy. Základní pojmy a způsoby reprezentace
a prohledávání; stromy a jejich reprezentace a prohledávání; planární 
grafy a jejich vlastnosti, algoritmy pro konstrukci minimální kostry, 
pro hledání nejkratší cesty; pro optimální párování; pro maximální 
toky atd.; datové struktury pro grafové algoritmy; náhodné grafy a 
jejich použití; náhodnostní grafové algoritmy.

\subsubsection*{Pojmy}
Graf (jednoduchý, orientovaný), cyklus, cesta, kompletný graf,
bipartitný graf, reprezentácia grafu, BFS, DFS, Best FS; strom,
koreňový strom, binárny vyhľadávací strom, in order, pre order,
post order, les; arc, slučka, planárny graf, izomorfizmus planárnych
grafov, rotačný systém, duálny graf, polygonálny planárny graf,
Eulerova formula, triangulácia, obmedzenie počtu hrán, Kuratowski theorem;
kostra, minimálna kostra, cut/cycle property, Kruskal, Jarník, 
Borůvka, Fredman-Tarjan, Gabow et al., Karger-Klein-Tarjan;
relaxácia hrany, Bellman-Ford, Dijkstra, Floyd-Warshall, Johnson;
matching (perfektné, maximálne), M-alternating/augmenting path,
bipartitný matching, blossom, Edmonds' blossom algorithm;
tok, reziduálna sieť, Ford-Fulkerson, Edmonds-Karp, level graph,
Dinic, MPM; minimálny rez, Gomory trees, globálny minimálny rez,
legal ordering, Naganochi-Ibaraki-Karger;
union-find, Fibonacciho halda; náhodné grafy a grafové algoritmy,
Kargerov algoritmus

\subsection{Základní pojmy, reprezentace a prohledávání}

Struktura $G = (V, E)$, kde $E \subseteq V^2$, se nazývá (neorientovaný) graf.
Pridaním orientácie k hranám (tj. $E \subseteq V \times V$) vznikne orientovaný graf.

Cesta $P_n$ délky $n$ je posloupnost $v_1,e_1,v_2,\ldots,v_n,e_n,v_{n+1}$ 
taková, že $e_i = (v_i, v_{i+1}) \in E$. Cyklus $C_n$ dĺžky $n$ vznikne
pridaním hrany $(v_n, v_1)$ k ceste dĺžky $n-1$. Cyklus dĺžky~$n$ má 
teda $n$~hrán a $n$~vrcholov, zatiaľ čo cesta dĺžky~$n$ má $n$~hrán a $n+1$~vrcholov.

Úplný graf $K_n$ má o veľkosti~$n$ má $n$~vrcholov, pričom každá dvojica
vrcholov je spojená hranou. V bipartitnom grafe sa vrcholy dajú
rozdeliť do dvoch množín $A,B$ tak, že $A \cap B = \emptyset$ a 
$A \cup B = V$, a pre každú hranu $(u,v)$ platí, že jeden z~jej
koncových vrcholov je v~množine $A$, zatiaľ čo druhý je v~množine $B$.

Graf reprezentuje reláciu $R \subseteq V \times V$ tak, že pokiaľ
$(a,b) \in R$, tak tieto vrcholy sú spojené hranou. Pokiaľ ide 
o~neorientovaný graf, táto relácia musí byť symetrická a antireflexívna.
Pri neorientovanom grafe nie sú na reláciu kladené žiadne obmedzenia.

V paměti graf reprezentujeme maticí (polem polí, hodí se pro hustý
graf), seznamem následníků (polem seznamů, hodí se pro řídký graf), či
implicitně nějakou funkcí (například graf markovovského rozhodovacího
procesu zadaný v~jazyce PRISM, hodí se pro velké grafy, kde stačí
prozkoumat malou část).

\bigskip
\noindent
\textbf{Použitá notace.} S každým grafem asociujeme množiny $V, E$,
často pro jednoduchost
$\forall v \in V \subseteq \mathbb{N} : 0 \leq v < \lvert V \rvert$.
Je-li graf ohodnocený, pak je navíc definována funkce $w : E \to
\mathbb{R}$. Pri analýze zložitosti občas používame $|V|=n$ a $|E|=m$.

\subsubsection*{Breadth-First Search}

Algoritmus prohledávání do šířky v následující variantě označí za
navštívené všechny vrcholy dosažitelné z počátečního vrcholu. V drobných
variantách slouží ke hledání nejkratší cesty nebo testování
dvojobarvitelnosti.

Paměťová náročnost je úměrná faktoru větvení, časová velikosti
$(\lvert V \rvert + \lvert E \rvert)$.

\begin{algorithm}
\caption{Breadth-First Search}
\begin{algorithmic}[1]
\Function{BFS}{G, r}
    \State Let $Q$ be an empty queue, push $r$ to $Q$.
    \State visited[$r$] = True
    \While{$Q$ is not empty}
        \State $u \gets Q$.dequeue()
        \For{$v \in$ neighbours of $u$}
            \If{not visited[$v$]}
                \State visited[$v$]$ \gets $True
                \State $Q$.enqueue($v$)
            \EndIf
        \EndFor
    \EndWhile
\EndFunction
\end{algorithmic}
\end{algorithm}

Tento algoritmus prehľadáva graf po~vrstvách, a keď sa dostane k~nejakému vrcholu,
tak rovno má najkrajšiu cestu k~nemu. Ako dátová štruktúra sa teda používa
fronta.

\subsubsection*{Depth-First Search}

Algoritmus prohledávání do hloubky v uvedené variantě označí za
navštívené všechny vrcholy dosažitelné z počátečního vrcholu. V drobných
variantách slouží ke hledání komponent grafu, topologickému uspořádání,
dále s většími úpravami například pro hledání silně souvislých komponent
(Tarjanův algoritmus).

Paměťová náročnost je úměrná hloubce grafu, časová velikosti
$(\lvert V \rvert + \lvert E \rvert)$.

Ako dátová štruktúra sa používa zásobník. Oproti BFS má menšiu pamäťovú 
náročnosť, keďže nemusí toľko vrcholov uchovávať v~zásobníku.
Vrchol je označovaný ako {\em biely}, keď jeho prieskum ešte nezačal,
{\em šedý}, ak jeho prieskum začal, ale ešte nebol dokončený, a {\em čierny},
ak jeho jeho prieskum bol dokončený (tj. neexistuje hrana, ktorá by z~neho
vychádzala a ešte by nebola preskúmaná).

\begin{algorithm}
\caption{Depth-First Search}
\begin{algorithmic}[1]
\Function{DFS}{G, u}
    \State visited[$u$] = True
    \For{$v \in$ neighbours of $u$}
        \If{not visited[$v$]}
            \State \Call{DFS}{G,v}
        \EndIf
    \EndFor
\EndFunction
\end{algorithmic}
\end{algorithm}

Pozor na \uv{nepravé} imperativní varianty DFS. Zejména by ke kontrole,
zda je vrchol navštívený, mělo docházet po vyndání ze zásobníku, ne při
vkládání.
%Někteří považují za pravou imperativní variantu pouze tu,
%která zachovává pořadí procházení následníků vrcholu
Problém vysvětluje Radek Pelánek ve své knize {\em Programátorská
cvičebnice}.

\subsection{Stromy, jejich reprezentace a prohledávání}

Nechť $G$ je graf. $G$ nazveme {\em strom}, pokud mezi každými dvěma
vrcholy existuje právě jedna cesta. Ekvivalentně je to souvislý graf
s~$\lvert V \rvert - 1$ hranami, prípadne súvislý graf bez~cyklov. 
Nesúvislý graf bez cyklov sa nazýva {\em les}. 

O~strome (resp. lese) hovoríme, že je {\em koreňový}, pokiaľ jeden 
z~jeho vrcholov (v~prípade lesu je to jeden z každej jeho komponenty)
je vybraný ako koreň. O~vrchole hovoríme, že je v hĺbke $l$, pokiaľ 
jeho vzdialenosť od koreňa je~$l$. Pri dvoch vrcholoch spojených
hranou hovoríme, že ten s nižšou hĺbkou je {\em rodičom} toho druhého.
Z~tranzitívneho uzáveru tejto relácie vznikajú pojmy {\em predok} a 
{\em potomok}.

Stromy v paměti často reprezentujeme jako datové struktury se seznamem
následníků. V~mnoha aplikacích potřebujeme ukládat ve vrcholech hodnoty.
Pro stromy s~hodnotami a s~pevnou aritou můžeme použít pole.
Například pro binární strom indexovaný od 0 máme na pozici~$0$ kořen,
každý vrchol na pozici~$i$ má levého potomka na indexu~$2i+1$, pravého
na indexu $2i+2$ a rodiče na indexu $\lfloor (i - 1) / 2 \rfloor$.

Stejně jako u obecných grafů můžeme použít DFS nebo BFS. Pro binární
stromy navíc rozlišujeme implementace DFS dle pořadí průchodu vrcholů:
pre-order, in-order, post-order. Prefix značí kdy je zpracován vrchol
relativně k~jeho potomkům.

\subsection{Planární grafy a jejich vlastnosti}

\begin{definition}
	{\em Oblúk} v rovine je spojitý obraz intervalu $[0,1]$ pod spojitým
	jednoduchým zobrazením $\varphi: [0,1] \to \mathbb{R}^2$. 
	{\em Smyčka} je oblúk, kde $\varphi(0)=\varphi(1)$.
\end{definition}
Neformálne je oblúk spojitou čiarou medzi dvomi bodmi takou, že sa
nikde nepretína. Smyčka sa nikde nepretína, ale má spoločný začiatočný
a konečný bod.

Graf je planární, pokud ho lze zakreslit do roviny tak, že se žádné dvě
hrany nekříží. Příklad $C_4$, nepříklad $K_{3,3}$ (lze na torusu).

{\em Rovinne nakreslený graf} je konkrétny nákres planárneho grafu v rovine.

\begin{definition}
	Pre rovinne nakreslený graf, nech $\Pi$ je množinou permutácií $\pi_v$
	pre každý vrchol $v$, kde $\pi_v$ je permutácia hrán incidentných 
	k~vrcholu~$v$ v~smere proti hodinovým ručičkám. $\Pi$~sa nazýva {\em rotačným
	systémom} nakresleného grafu.
	
	Dve nakreslenia sú {\em ekvivalentné}, pokiaľ medzi nimi existuje izomorfizmus,
	ktorý zachováva rotačný systém, až na možný zrkadlový obraz (kde všetky permutácie
	sú opačné).
\end{definition}

\begin{definition}[Stena]
	V rovinne nakreslenom grafe, oblúkovo súvislé regióny roviny (vrátane vonkajšieho),
	rozdelené nákresom grafu, sa nazývajú stenami grafu.
	
	Hranice každej steny tvoria uzatvorený sled\footnote{walk}, ktorý sa pre
	danú stenu nazýva stenovou prechádzkou. Jej dĺžka sa nazýva veľkosťou steny.
\end{definition}

\begin{definition}[Duálny multigraf]
	Pre graf $G=(V,E,\epsilon)$ je graf $H=(F,E,\phi)$ {\em duálnym multigrafom}, pokiaľ
	\begin{itemize}
		\item $F$ je množina stien v $G$,
		\item $\phi$ je reláciou, ktorá pre každú $e \in E$ spája steny, ktoré táto hrana rozdeľuje.
	\end{itemize}
	$G$ sa v tomto prípade nazýva {\em primálnym grafom} multigrafu $H$.
\end{definition}

\begin{definition}[Polygonálny planárny rovinne nakreslený graf]
	{\em Lomená čiara} je konečná postupnosť bodov $(p_0, p_1, \ldots,
	p_k) \subseteq \mathbb{R}^2$ spolu s rovnými čiarami $p_0p_1, p_1p_2, 
	\ldots, p_{k-1}p_k$ takými, že žiadne dve nemajú spoločný prienik
	(okrem možného prieniku keď $p_0=p_k$ v tomto bode).
	
	{\em Polygonálny planárny rovinne nakreslený graf} je planárne rovinne
	nakreslený graf, ktorého všetky hrany sú lomené čiary.
\end{definition}

Pre každý planárny rovinne nakreslený graf existuje ekvivalentný polygonálny
rovinne nakreslený graf.

Pokiaľ sa v stenovej prechádzke jedna hrana nachádza dva krát, táto hrana
sa nazýva {\em mostom}. Vrchol, ktorý sa v nej nachádza dva krát sa nazýva
{\em cutvertex}.

\begin{theorem}[Kuratowski]
Graf je planární právě tehdy, když nemá jako podgraf
podrozdělení\footnote{G je podrozdělením H pokud lze opakovaným
rozdělováním hran na dvě (vložením vrcholu doprostřed) získat z H graf
G.} $K_5$ nebo $K_{3,3}$.
\end{theorem}

Dále uvažujme zakreslení grafů bez křížení v~rovině a s~$f$
stěnami.

\begin{theorem}[Euler]
Pre planárny spojitý graf platí 
$v - e + f = 2$.
\end{theorem}

\begin{proof}
Zakresli $G$ bez křížení.  Dokud
$G$ není strom, odebírej hranu nějakého cyklu, sníží se $e$ i $f$ o
jedna; nakonec dostaneme strom s~$v = e + 1, f = 1$, kde $v - e + f = 2$
jistě platí, obrácením postupu je důkaz hotov.
\end{proof}

\begin{corollary}
$e \leq 3v - 6$.
\end{corollary}

\begin{proof}
Máme $e = v + f - 2$, tedy $3e = 3v + 3f -
6$, stačí si všimnout, že každá stěna je ohraničena nejméně třemi
hranami a každá hrana se dotýká dvou stěn, tedy $3f \leq 2e$,
a získáme $e \leq 3v - 6$.
\end{proof}

\begin{note}
    Předpoklad rovinného zakreslení je důležitý,
    například na torusu platí $v - e + f = 0$.
\end{note}

\subsection{Konstrukce minimální kostry}

Kostra $T$ grafu $G$ je spanning podgrafom\footnote{Spanning podgraf $T$
obsahuje všetky vrcholy pôvodného grafu, tj. $V(T)=V(G)$.} grafu $G$ takým,
že $T$~je stromom. Minimálna kostra pri ohodnotenom grafe je takou kostrou,
že súčet hodnôt hrán, ktoré obsahuje, je minimálny.

\begin{theorem}[Cut property]
	Pokiaľ $e$ je nahľahšou v nejakom reze grafu $G$, táto hrana musí
	byť v minimálnej kostre.
\end{theorem}

\begin{theorem}[Cycle property]
	Pokiaľ $e$ je najťažšou v nejakom cykle grafu $G$, táto hrana nemôže
	byť v minimálnej kostre.
\end{theorem}

\subsubsection*{Kruskalův algoritmus}

Ukážeme Kruskalův algoritmus. Přidává hrany v pořadí od nejlehčí tak,
aby nikdy nevytvořil cyklus.
Algoritmus udržuje informaci o komponentách grafu v datové struktuře
Union Find, která je je popsána v
\autoref{subsec:data_structures}.

\begin{algorithm}
\caption{Kruskal}
\begin{algorithmic}[1]
\Function{Kruskal}{$G$}
    \State $A = \emptyset$
    \State $uf \gets UnionFind(V(G))$
    \For{$(u,v) \gets E(G)\text{ sorted by }w$, increasing}
        \If{$uf.find(u) \neq uf.find(v)$}
            \State $A = A \cup \{(u,v)\}$
            \State $uf.union(u, v)$
        \EndIf
    \EndFor
    \State \Return $A$
\EndFunction
\end{algorithmic}
\end{algorithm}

Zložitosť Kruskalovho algoritmu je $\mathcal{O}(m \log(m))=\mathcal{O}(m \log(n))$, pričom
časovo najnáročnejšie je zoradenie hrán.

\subsubsection*{Jarníkov algoritmus}

V Jarníkovom (niekedy nazývaným aj Primovom) algoritme sa začne
náhodným výberom jedného vrcholu z grafu a jeho vložením do kostry.
Algoritmus pokračuje tak, že vyberie najľahšiu hranu opúšťajúcu
aktuálnu kostru, a pridá daný vrchol do kostry. Toto sa opakuje,
dokým kostra neobsahuje všetky vrcholy grafu.

Zložitosť Jarníkovho algoritmu pomocou Fibonacciho haldy
 je $\mathcal{O}(m + n \log n)$.

\subsubsection*{Borůvkov algoritmus}

Borůvkov algoritmus začne tým, že graf berie ako les obsahujúci
$n$ jedno-vrcholových komponent. Pokračuje sa tak, že pre každú
komponentu sa nájde najľahšia hrana, ktorá ju opúšťa, a komponenty
spájané touto hranou sa spoja. Opakuje sa, dokým neostane jediná 
komponenta, ktorá bude minimálnou kostrou.

V Borůvkovom algoritme sa vykoná najviac $\log(n)$~fáz, pričom
každá fáza sa vykoná v čase $\mathcal{O}(m)$. Celková zložitosť
je teda $\mathcal{O}(m \log(n))$.

\subsubsection*{Fredman-Tarjan}

V tomto algoritme sa simuluje Jarníkov algoritmus s~haldou s~obmedzenou
veľkosťou najviac~$k$. Keď sa táto veľkosť prekoná, náš budovaný strom má 
viac ako $k$~susedov. Vtedy začneme budovať kostru z nového vrcholu. Týmto
spôsobom dostaneme les, ktorý stiahneme\footnote{contract; nie som si istý 
prekladom. Princípom je, že z každého stromu sa stane v novom grafe jeden
vrchol.} a znova spustíme algoritmus.

Zložitosť Fredman-Tarjan algoritmu je $\mathcal{O}(m \cdot \beta(m,n))$, kde
funkcia $\beta(m,n)$ je definovaná ako $\beta(m,n)= \min\{ i | \log^{(i)}n \leq m/n \}$.
Tento algoritmus je v každom prípade aspoň tak rýchly, ako Jarník, pričom
je lepší najmä v~prípade riedkych grafov.

\subsubsection*{Gabow et al.}

Základnou myšlienkou tohto algoritmu je, že sa vytvorí $2m/p$ paketov,
kde každý z~nich obsahuje $p$ hrán. Každý paket má váhu svojej najľahšej
hrany. Potom sa simuluje Fredman-Tarjan algoritmus, ktorý používa pakety
miesto hrán.

Za použitia Fibonacciho haldy je zložitosť tohto algoritmu 
$\mathcal{O}(m \cdot \log\beta(m,n))$.

\subsubsection*{Karger-Klein-Tarjan}

Tento algoritmus využíva princíp $F$-ľahkých a $F$-ťažkých hrán.
Nech $F\subseteq G$ je podgraf grafu $G$, ktorý je lesom. Pre
každú hranu $e=(u,v) \in E(G)$ nech $w_F(e)$ je váha najťažšej hrany
na ceste medzi vrcholmi $u,v$ v lese $F$, prípadne $\infty$ pokiaľ
dané vrcholy v lese $F$ nie sú v rovnakej komponente. Hrana $e$
sa nazýva {\em $F$-ťažkou}, pokiaľ jej váha je väčšia ako $w_F(e)$, 
{\em $F$-ľahkou} inak.

Pre pochopenie tohto algoritmu najskôr ukážeme prvý pokus o~náhodnostný
algoritmus, na ktorom budeme ilustrovať kocepty, ktoré sa potom 
použíjú v~algoritme Karger-Klein-Tarjan. Začneme tak, že z grafu $G$ náhodne
vytvoríme pomocný graf $H$, v~ktorom je každá hrana s~pravdepodobnosťou~$1/2$. 
V grafe~$H$ rekurzívne nájdeme minimálnu kostru $F$ (aj keď v tomto prípade
táto kostra nemusí byť spojitá, keďže graf $H$ nemusí byť spojitý).
V kostre $F$ nájdeme všetky $F$-ťažké hrany, tieto v minimálej 
kostre grafu $G$ nemôžu byť, kvôli cycle property. Tieto $F$-ťažké
hrany teda z grafu $F$ odobereme. Na zvyšku tohto grafu~$F$ budeme
opäť rekurzívne hľadať minimálnu kostru a tento postup budeme opakovať, kým
nedostaneme minimálnu kostru grafu $G$.

\begin{algorithm}
\caption{Karger-Klein-Tarjan}
\begin{algorithmic}[1]
\Function{RandomizedMSF}{$G$}
    \State \textbf{if} {$E = \emptyset$} \textbf{then} \Return $\emptyset$
    \State $(G_1, F_1) \gets \text{BoruvkaPhase}(G)$
	\State $(G_2, F_2) \gets \text{BoruvkaPhase}(G_1)$
	\State $H \gets \text{náhodný podgraf }G_2\text{, kde každá hrana je s pravdepodobnosťou }1/2$
	\State $F' \gets RandomizedMSF(H)$
	\State $E' \gets F'\text{-ľahké hrany z }G_2$
	\State $H' \gets G[E']$
	\State $F_3 \gets RandomizedMSF(H')$
    \State \Return $F_1 \cup F_2 \cup F_3$
\EndFunction
\end{algorithmic}
\end{algorithm}

Zložitosť Karger-Klein-Tarjan algoritmu je $\mathcal{O}(m+n)$.

\subsection{Hledání nejkratší cesty}

V probléme hľadania najkratšej cesty ide o to nájsť takú cestu
medzi vrcholmi, že súčet váh hrán na nej je minimálny. V prípade
neohodnotených grafov predpokladáme, že váha každej hrany je~$1$.

Existujú rôzne varianty tohto problému, a to najmä hľadanie najkratšej
cesty medzi dvomi konkrétnymi vrcholmi, hľadanie najkratšej cesty
z~konkrétneho vrcholu do všetkých ostatných, a hľadanie najkratšej
cesty medzi všetkými dvojicami vrcholov.

Algoritmy pre tento problém často využívajú pojem {\em relaxácia hrany}.
Majme hranu $e=(u,v)$ s váhou $w(e)$ a pre vrcholy $u,v$ majme nejakú
ich vzdialenosť $d(u), d(v)$ od vrcholu~$s$. Potom relaxácia tejto
hrany je proces, kde skúsime, či sa pomocou tejto hrany dá
vylepšiť minimálna vzdialenosť vrcholu $v$ od $s$, tj. nastavíme
$d(v)$ na $min(d(v),d(u)+w(e))$.

\subsubsection*{Bellman-Ford}
Tento algoritmus počíta najkrašiu vzdialenosť z jedného vrcholu
(nazvime ho~$s$) do všetkých ostatných. Na rozdiel od Dijkstrovho
algoritmu funguje aj na grafoch so záporne hodnotenými hranami.

V jednej iterácii Bellman-Ford algoritmus relaxuje každú hranu.
Takúto iteráciu algoritmus opakuje $n$ krát, a tým končí. 

Nakoniec ešte posledný krát zopakuje relaxáciu všetkých hrán,
a pokiaľ je ešte nejaká relaxácia úspešná, graf obsahuje 
záporný cyklus a algoritmus končí s chybou.

Zložitosť tohto algoritmu sa dá triviálne odvodiť a je to $\mathcal{O}(mn)$.

\subsubsection*{Dijkstra}
Dijkstrův algoritmus udržuje jednoduchý invariant:
Pro každý navštívený vrchol~$v$, je $dist[v]$ nejkratší cesta z $s$ do
$v$. Pro každý nenavštívený vrchol $u$, je $dist[u]$ nejkratší cesta z
$s$ do $u$ přes dosud navštívené vrcholy.

\begin{algorithm}[h]
\caption{Dijkstra}
\begin{algorithmic}[1]
\Function{Dijkstra}{$G, s, t$}
    \State $Q \gets V(G)$
    \State $\forall v \in V : dist[v] \gets \infty, prev[v] \gets None$
    \State $dist[s] \gets 0$

    \While{$Q$ is not empty}
        \State $u \gets \argmin_{u \in Q} dist[u]$
        \State remove $u$ from $Q$
        \For{$v \in$ neighbours of $u$}
            \If{$dist[u] + w(u,v) < dist[v]$}
                \State $dist[v] \gets dist[u] + w(u,v) $
                \State $prev[v] \gets u$
            \EndIf
        \EndFor
    \EndWhile
    \State \Return (dist, prev)
\EndFunction
\end{algorithmic}
\end{algorithm}

Za použitia Fibonacciho haldy je zložitosť algoritmu $\mathcal{O}(m + n \log n)$

\subsubsection*{Floyd-Warshall}

Algoritmus Floyd-Warshall počíta najkrašiu cestu medzi všetkými
dvojicami vrcholov.

Floydův-Warshallův algoritmus postupně napočítává nejkratší cestu z $i$
do $j$, vždy zváží cestu přes vrchol $k$, takže v $k$-tém kroce
má správnou cestou pomocí prvních $k$ vrcholů.
Rekonstrukce cesty se dá provést z~proměnné $next$, inicializované
$next[u][v] \gets v$ a dále nastavované $next[i][j] \gets next[i][k]$
při nalezení kratší cesty.
Floyd-Warshallův algoritmus se dá také použít k detekci záporných cyklů.

\begin{algorithm}[H]
\caption{Floyd-Warshall}
\begin{algorithmic}[1]
\Function{FloydWarshall}{$G, s, t$}
    \State $\forall (u,v) \in V^2 : dist[u][v] \gets 0$
        \textbf{if} $u = v$ \textbf{else} $\infty$
    \State $\forall (u,v) \in E : dist[u][v] \gets w(u,v)$
    \For{$k$\textbf{ in }$range(\lvert V \rvert)$}
        \For{$i$\textbf{ in }$range(\lvert V \rvert)$}
            \For{$j$\textbf{ in }$range(\lvert V \rvert)$}
                %\If{$dist[i][k] + dist[k][j] < dist[i][j]$}
                    %\State $dist[i][j] \gets dist[i][k] + dist[k][j]$
                %\EndIf
                \State $dist[i][j] \gets \min(dist[i][j], dist[i][k] + dist[k][j])$
            \EndFor
        \EndFor
    \EndFor
\EndFunction
\end{algorithmic}
\end{algorithm}

V každej z $n$~iterácií pre každú dvojicu vrcholov skúsi,
či sa ich vzdialenosť dá znížiť cez práve prechádzaný vrchol. 
Zložitosť algoritmu je teda $\mathcal{O}(n^3)$.

\subsubsection*{Johnson}
Tento algoritmus hľadá najkrajšiu cestu medzi všetkými dvojicami vrcholov.
V tomto algoritme skúsime nájsť záporný cyklus - v tom prípade
končíme s chybou. Inak modifikujeme hrany tak, aby už neobsahovali
záporné hrany, a spustíme Dijkstru z každého vrcholu.

Zložitosť tohto algoritmu je súčet zložitosti Bellman-Ford (pre detekciu
záporných cyklov) + $n$ krát Dijkstra. Teda celková zložitosť
je $\mathcal{O}(n^2 \log(n) + mn)$.

\subsubsection*{Algoritmus A*}
Algoritmus A* je modifikáciou Dijkstrovho algoritmu s~tým,
že váhy hrán sa modifikujú, aby ovplyvnili rýchlosť výpočtu.
Napríklad v mapovej navigácii sa do váhy zakomponuje Euklidovská
vzdialenosť od cieľa, aby sa cesty, ktoré vedú správnym smerom,
vyhodnocovali s~väčšou váhou, ako cesty, ktoré idú opačne.

Nech váha hrany $e=(u,v)$ je $w(e)$. Potom modifikovaná
váha hrany bude $w'(e) = w(e) + p(v) - p(u)$, kde $p(x)$
pre vrchol $x$ je potenciálová funkcia pre daný vrchol.

\subsection{Optimální párování}

% https://en.wikipedia.org/wiki/Matching_(graph_theory)#Maximal_matchings
% https://ksp.mff.cuni.cz/encyklopedie/parovani.html
% https://is.muni.cz/auth/el/1433/podzim2015/MA015/um/04-matchings.pdf

\begin{definition}
    {\em Párování} $M \subseteq E$ grafu $G$ jsou hrany takové, že
    žádné dvě nesdílí vrchol.

    Párování je {\em maximální}, pokud je maximální vzhledem k inkluzi.

    Párování je {\em největší}, pokud má nejvyšší možnou velikost.

    Párování je {\em perfektní} pokud pokrývá všechny vrcholy.
\end{definition}

Problém maximálního párování je triviálně řešitelný hladovým
algoritmem.

Problém hledání největšího párování v bipartitním grafu je řešitelný
jednoduchým algoritmem na bázi toků. Mějme bipartitní graf
$G = (A \cup B, E)$. Přidáme vrcholy $s, t$,
hrany $(s, a), a \in A, (b, t), b \in B$ a každé hraně přiřadíme
kapacitu $1$. Hrany maximálního $s-t$ toku tvoří maximální párování.

%Alternativní algoritmus opakovaně hledá pomocí DFS {\em zlepšující
%cestu}\footnote{{\em Zlepšující cesta} pro dané párování je taková, ve které
%se po řadě alternuje příslušnost hran do párování a navíc první a
%poslední hrana v párování nejsou.}, když ji najde, tak příslušnost hran do
%párování prohodí a proces opakuje. Když ji nenajde skončí, protože
%nalezl maximální párování\footnote{Důkaz:
%\href{https://ksp.mff.cuni.cz/encyklopedie/parovani.html}{https://ksp.mff.cuni.cz/encyklopedie/parovani.html}}.

Problém největšího párování v~obecných grafech řeší Edmondsův blossom
algoritmus, jehož myšlenky pouze nastíníme.
Algoritmus opakovaně hledá pomocí {\em zlepšující
cestu} (cesta ve které
se po řadě alternuje příslušnost hran do párování a navíc první a
poslední hrana v párování nejsou), když ji najde, tak příslušnost hran do
párování prohodí a proces opakuje. Když ji nenajde skončí, protože
nalezl maximální párování.

Nyní stručně popíšeme jak funguje algoritmus pro hledání zlepšujících
cest. Vrchol nazveme {\em odhalený}, pokud žádná přilehlá hrana není v~párování.
Nejprve založí strom $F$, do kterého přidá všechny vrcholy odhalené
v~$M$. Každý strom se potom nějakým (?) průchodem buduje, až je v~něm nalezena zlepšující
cesta (ta je vrácena), blossom (cyklus s~$2k+1$ vrcholy, z~nichž $k$ je
v~párování; ten je kontrahován a algoritmus pokračuje), případně ani
jedno a potom zlepšující cesta neexistuje.

\begin{figure}[H]
    \centering
    \includegraphics[width=300pt]{blossom_contraction.png}
    \caption{Kontrakce květu, \href{https://en.wikipedia.org/wiki/Blossom_algorithm}{zdroj}}
\end{figure}

\subsection{Maximální tok}
Máme orientovaný graf, kde každá hrana má priradenú kapacitu.
Naším cieľom je vypočítať maximálny tok, ktorý môže
grafom prúdiť z vrcholu $s$ do vrcholu~$t$.

V~průběhu algoritmu udržujeme čtyři invarianty:
$f(u,v) = -f(v,u)$, tok nepřevyšuje kapacitu,
do každého vrcholu (kromě $s,t$) vteče stejně jako z~něj vyteče,
kolik vyteče z~$s$, tolik doteče do $t$.

Funkce $c : E \to \mathbb{R}$ přiřazuje každé hraně její maximální
možnou kapacitu. Definujeme {\em reziduální graf}
$G_{f}(V,E_{f})$ s~kapacitou $c_{f}(u,v)=c(u,v)-f(u,v)$ a bez toku.
Reziduálny graf teda obsahuje hrany s ohodnoteniami, ako sa tok
cez danú hranu môže zmeniť. Keď je napr. v sieti hrana $(x,y)$
s naplnenou kapacitou $8/10$, tak v reziduálnom grafe je
hrana $(x,y)$ s kapacitou $2$ a hrana $(y,x)$ s kapacitou $8$.

{\em Augmentujúca cesta} je ľubovoľná cesta z $s$ do $t$
v~reziduálnom grafe. {\em Bottleneck kapacita} je augmentujúcej
cesty je minimálna kapacita z~jej hrán.

\begin{theorem}
Hodnota maximálního toku je zároveň váhou minimálního řezu.
\end{theorem}

\subsubsection*{Ford-Fulkerson}
Algoritmus udržiava reziduálny graf. Dokým v~tomto grafe
existuje augmentujúca cesta, tak nejakú nájdeme a 
augmentujeme ju\footnote{Pri augmentovaní cesty ju naplníme 
a aktualizujeme reziduálny graf}.

Časová zložitosť jednej iterácie je $\mathcal{O}(m)$, pretože
toľko trvá nájdenie augmentujúcej cesty. Predpokladáme, že 
kapacity hrán sú celočíselné, inak by nebolo zaručené, že algoritmus skončí. 
Nech $C$ je maximálna kapacita. Zložitosť algoritmu je potom $\mathcal{O}(nm \cdot C)$.

\subsubsection*{Edmonds-Karp}

Tento algoritmus je takou modifikáciou Ford-Fulkerson algoritmu,
že v každej iterácii vyberie takú augmentujúcu cestu, ktorá obsahuje
minimálny počet hrán.

Hledání cesty na řádku 3 se provede algoritmem BFS. S DFS by algoritmus
neměl zaručenou konvergenci. Pro cesty délky $k$
může být až $m$ zlepšujících cest, délka cesty je nejvýše
$n$, celkem $mn$
počtů nalezení zlepšujících čas a zlepšení trvá $m$
kroků -- složitost algoritmu je
$\mathcal{O}(nm^2)$.

\begin{algorithm}[H]
\caption{Edmonds-Karp}
\begin{algorithmic}[1]
\Function{EdmondsKarp}{$G, s, t, c$}
    \State $f(u,v) \gets 0$ for all $(u,v) \in E$
    \While{exists $s$--$t$ path $p$ in $G_f$, s.t. $c_f(u,v) > 0$ for all $(u,v) \in p$}
        \State $c_f(p) = \min \{ c_f(u,v) \mid (u,v) \in p \}$
        \For{$(u,v) \in p$}
            \State $f(u,v) \gets f(u,v) + c_f(p)$
            \State $f(v,u) \gets f(v,u) - c_f(p)$
        \EndFor
    \EndWhile
\EndFunction
\end{algorithmic}
\end{algorithm}

\subsubsection*{Dinic}
Dinicov algoritmus využíva dva pojmy, ktoré si najprv definujeme.

\begin{definition}[Level graf]
Nech $L_i$ je množina vrcholov, ktoré sú vo vzdialenosti~$i$ od
vrcholu $s$. {\em Level graf} $L_G$ pre graf $G$ je jeho podgrafom
takým, že obsahuje iba hrany $(u,v)$ také, že $u \in L_i$ a 
$v \in L_{i+1}$ pre nejaké prirodzené číslo~$i$.
\end{definition}

Pre graf $G$ sa jeho level graf dá vypočítať v čase $\mathcal{O}(m)$.

\begin{definition}[Blokujúci prúd]
Hovoríme, že prúd je {\em blokujúci}, pokiaľ každá $s-t$ cesta v~danom
prúde obsahuje aspoň jednu blokujúcu hranu. 
\end{definition}

\begin{lemma}
	Každý maximálny prúd je blokujúci, avšak nie každý blokujúci prúd
	je maximálny.
\end{lemma}

Myšlienkou Dinicovho algoritmu je to, že všetky jeho minimálne augmentujúce
cesty augmentujeme naraz. Keďže dĺžka najkratšej augmentujúcej cesty
nemôže klesnúť, toto nám zaručí najviac $n$ iterácií.

V každej iterácii musíme nájsť level graf aktuálneho reziduálneho grafu,
v ňom nájsť blokujúci prúd, tento potom augmentovať, a nakoniec 
aktualizovať reziduálny graf.

Každá iterácia sa dá implementovať so zložitosťou $\mathcal{O}(mn)$, takže
zložitosť celého algoritmu je $\mathcal{O}(mn^2)$.

\subsubsection*{MPM algoritmus}
Modifikácia Dinicovho algoritmu, ktorá každú iteráciu vykoná v čase $\mathcal{O}(n^2)$.
Jeho zložitosť je teda $\mathcal{O}(n^3)$.

Popis TODO

\subsubsection*{Push-relabel}
Tento algoritmus bude využívať pseudotok, ktorý spĺňa kapacitné
ohraničenie, avšak tok prichádzajúci do vrcholu môže byť väčší, 
ako tok odchádzajúci z tohto vrcholu (tj. vrchol môže určitú časť
toku ukladať do rezervnej nádrže). Každý vrchol má v priebehu 
algoritmu svoju výšku, ktorá sa iba zväčšuje. Zdroj má fixovanú
výšku $n$, stok má fixovanú výšku $0$, ostatné vrcholy majú na začiatku
algoritmu výšku $0$.

Na začiatku algoritmu sa zo zdroja pretlačí maximálny tok
do všetkých možných hrán. Tok, ktorý pritečie do vnútorného vrcholu
siete sa uloží do rezervnej nádrže; odtiaľ sa potom postupne pretláča
do vrcholov s menšou výškou (\textsc{Push}). Keď nastane situácia, že 
rezervná nádrž nie je prázdna, ale všetky nenasýtené hrany, ktoré z~vrchola
odchádzajú, vedú do vrcholov s~väčšou výškou, zvýšime výšku vrchola
(\textsc{Relabel}). Výpočet končí, keď už nie je možné pretlačiť viac
toku do stoku.

Pokiaľ sa v priebehu algoritmu vyberá ako aktívny vrchol vždy ten
s~maximálnou výškou, zložitosť algoritmu je $\mathcal{O}(n^3)$.

\subsection{Datové struktury pro grafové algoritmy}
\label{subsec:data_structures}

\subsubsection*{Union Find}
Pro implementaci Kruskalova
algoritmu používáme strukturu Union Find.

Struktura \textsc{Union Find} ukládá vrcholy do stromů.
Dva vrcholy jsou ve stejné komponentě právě tehdy, když jsou oba
ve stejném stromu, což poznáme pomocí porovnání kořenů (\textsc{Find}).
Spojení komponent probíhá přidáním jednoho stromu k druhému -- to se
řídí podle hodnoty $rank$ každého stromu tak, aby hloubka zůstala nízká.

V \autoref{alg:union_find} proměnnou $self$ používáme ve smyslu jako v
Pythonu. Funkce \textsc{Initialize} je konstruktor.

\begin{algorithm}[H]
\caption{Union Find}
\label{alg:union_find}
\begin{algorithmic}[1]
\Function{Initialize}{$self, V$}
    \State $\forall v \in V : self.parent[v] \gets v, self.rank[v] \gets 0$
\EndFunction

\Function{Find}{$self, v$}
    \While{$self.parent[v] \neq v$}
        \State $v \gets self.parent[v]$
    \EndWhile
    \State \Return $v$
\EndFunction

\Function{Union}{$self, u, v$}
    \State $ru \gets \Call{Find}{u}, rv \gets \Call{Find}{v}$
    \Comment{Najdi kořeny stromů.}
    \If{$ru \neq rv$}
        \If{$self.rank[ru] > self.rank[rv]$}
            \State $self.parent[rv] \gets ru$
        \ElsIf{$self.rank[ru] < self.rank[rv]$}
            \State $self.parent[ru] \gets rv$
        \Else
            \State $self.parent[rv] \gets ru$
            \State $self.rank[ru] \gets self.rank[ru] + 1$
        \EndIf
    \EndIf
\EndFunction

\end{algorithmic}
\end{algorithm}

\subsubsection*{Fibonacciho halda}

Pre implementáciu viacerých algoritmov sa využíva Fibonacciho halda.
Fibonacciho halda je modifikáciou binomiálnej haldy, ktorá umožňuje
efektívne realizovať operácie union, insert a decrease-key.
Keďže táto halda je pomerne zložitá, jej popis popíšeme iba zjednodušene.

Každý vrchol má v sebe uložený kľúč, binárny príznak, počet synov
a ukazetele na otca, ľavého a pravého brata a jedného zo synov.
Halda sama o sebe obsahuje počet prvkov a ukazateľ na najmenší z nich.

Pri vkladaní sa vrchol iba vloží na prvú úroveň (do koreňového zoznamu), 
a pokiaľ je jeho kľúč menší ako prechádzajúce minimum, ukazateľ
na minimum sa nastaví na nový vrchol. Zložitosť tejto operácie je konštantná.

Nájdenie minimálneho prvku je triviálne a konštantné.

Pri zjendnotení dvoch háld proste spojíme koreňové zoznamy. Zložitosť
tejto operácie je teda opäť konštantná.

Pri odstránení minimálneho prvku všetkých jeho synov presunieme do 
koreňového zoznamu. Pomocou funkcie \textsc{Consolidate} upravíme haldu,
aby neobsahovala dva stromy rovnakého stupňa. Nový minimálny 
prvok nájdeme prechádzaním koreňového zoznamu.

Funkcia \textsc{Consolidate} nájde dva vrcholy $x,y$ s rovnakým stupňom,
$x.key \leq y.key$, a urobí $y$ synom vrcholu $x$. Táto operácia
sa robí opakovane.

Amortizovaná zložitosť odstránenia minimálneho prvku je logaritická.

Ak sa pri znížovaní kľúča poruší vlastnosť minimovej haldy,
celý podstrom sa odreže a presunie sa do koreňového zoznamu.
Pokiaľ sa nejakému vrcholu odreže druhý syn, tak aj tento vrchol
sa presunie do koreňového zoznamu. Toto môže viezť ku kaskádovému
odrezávaniu smerom ku koreňu.

Amortizovaná zložitosť znižovania kľúča je opäť konštantná.



\subsection{Náhodné grafy a jejich použití}

Ukážeme příklad z~předmětu IA062, kde se pomocí {\em probabilistic
method} dokáže, že nějaká vlastnost v~náhodném grafu platí a z~toho se
potom odvodí náhodnostní algoritmus.

% https://www.corelab.ntua.gr/courses/netalg/presentations/presentations2015/maxcut.pdf

\begin{definition}
    Mějme graf $G$. Problém {\em maximálního řezu} je rozdělit $V$ do
    množin $A, B$ tak, aby $\lvert \{ (u,v) \mid u \in A, v \in B \}
    \rvert$ byla maximální.
\end{definition}

\begin{theorem}
    Pro každý graf $G$ existuje maximální řez velikosti alespoň
    $\lvert E \rvert / 2$.
\end{theorem}

\begin{proof}
    Každý vrchol je s~nezávislou a uniformní pravděpodobností přiřazen
    do jedné z~množin $A, B$. Pravděpodobnost, že hrana je mezi $A, B$
    je tak $1/2$. Podle linearity očekávané hodnoty je velikost řezu
    $\lvert E \rvert / 2$. Z~toho vyplývá, že musí existovat přiřazení
    splňující tvrzení věty\footnote{Tady se právě uplatní myšlenka
    probabilistic method -- aby očekávaná hodnota byla $x$, musí
    existovat nenulová pravděpodobnost, že hodnota je $\geq x$.}.
\end{proof}

\begin{theorem}
    Existuje Las Vegas algoritmus pro hledání řezu velikosti alespoň
    $m/2$.
\end{theorem}

\begin{proof}
    Algoritmus náhodně rozdělí vrcholy do $A, B$, označme velikost řezu
    $s$ a označme
    \[
    p = P(s > \frac{m}{2})
    \]
    Nyní vypočteme
    \[
        \frac{m}{2}
        = \Expect(s)
        = \sum_{i < m/2} i P(s = i)
        + \sum_{i \geq m/2} i P(s = i)
        \leq (1-p)(\frac{m}{2} - 1) + mp
    \]
    což nám dává pravděpodobnost úspěchu
    \[
        p \geq \frac{1}{m/2 + 1}
    \]
    a tedy očekávaný počet opakování náhodného rozdělení je $m/2 + 1$
    (dle věty o očekávané hodnotě geometrického rozdělení).
    Ověření velikosti řezu má složitost nejvýše $m$, celkem tedy máme
    Las Vegas algoritmus s~očekávanou složitostí $\mathcal{O}(m^2)$.
\end{proof}

\subsection{Náhodnostní grafové algoritmy (Kargerův)}

\begin{definition}[Nejmenší řez]
    {\em Řez} $(S,T)$ je rozklad $V$.
    {\em Velikost řezu} $(S,T)$ je
    $w(S,T) = \lvert \{ (u,v) \in E \mid u \in S, v \in T \} \rvert$.
    Řez $(S,T)$ je {\em nejmenší} pokud $w(S,T)$ je nejmenší.
\end{definition}

Problém hledání nejmenšího řezu řeší Kargerův náhodnostní algoritmus.
V grafu vybírá náhodně hrany a kontrahuje je, dokud nezbydou poslední
dva vrcholy. Hrany mezi nimi tvoří vrácený řez.
Kontrakci hrany lze provést v čase $\mathcal{O}(n)$ a provede se jich
$n$, celková složitost je $\mathcal{O}(n^2)$.

\begin{algorithm}
\caption{Kargerův algoritmus}
\label{alg:karger}
\begin{algorithmic}[1]
\Function{Karger}{$G = (V, E)$}
    \While{$\lvert V \rvert \geq 2$}
        \State choose $e \in E$ uniformly at random
        \State contract $e$ in $G$
    \EndWhile
    \State \Return the unique cut of $G$
\EndFunction
\end{algorithmic}
\end{algorithm}

Zbývá analyzovat pravděpodobnost chyby. Nechť $C$ je nějaký zvolený
nejmenší řez s~$k$ hranami. Minimální stupeň v $G$ musí být tedy nejméně
$k$. Pravděpodobnost, že nějaká hrana z~$C$ byla vybrána při první
kontrakci je tedy
\[
    \frac{k}{\lvert E \rvert} \leq \frac{k}{\frac{nk}{2}} = \frac{2}{n}
\]
Při $(i+1)$. kontrakci je to nejvýše $\frac{2}{n-i}$. Pravděpodobnost,
že žádná z~hran $C$ nebyla vybrána (a tedy $C$ je vrácen) je
\[
    \prod_{i = 0}^{n - 3} (1 - \frac{2}{n-i}) =
    \prod_{i = 0}^{n - 3} (\frac{n - i - 2}{n-i}) =
    {n \choose 2}^{-1} =
    \frac{2}{n(n-1)}
\]
Nyní můžeme algoritmus $kn^2$-krát opakovat a získáme následující
horní odhad na pravděpodobnost chyby
(použijeme $1 - x \leq e^{-k}$).
\[
    \big (1 - \frac{2}{n(n-1)} \big )^{kn^2}
    \leq \big (1 - \frac{2}{n^2} \big )^{kn^2}
    \leq \big (e^{-\frac{2}{n^2}} \big )^{kn^2}
    = e^{-2k}
\]
Při $(\log n) n^2$ opakováních tedy dostáváme pravděpodbnost chyby
pouze $\approx \frac{1}{n^2}$. Složitost je $n^4 \log n$.
